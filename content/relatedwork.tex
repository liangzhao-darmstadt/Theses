\chapter{Related Work}
\label{cha:RelatedWork}

\section{Distributed Differential Privacy (DDP)}
\label{sec:DistributedDifferentialPrivacy}
% To achieve differential privacy (DP) in a federated learning scenario, either a central trusted server (centralized model) is assumed to perform perturbation to the aggregated result, or each client perturbs its data (local model) before sending it to the central server. However, in the centralized model, the trusted server is responsible for security and privacy and becomes the single point of failure for the entire system. In the decentralized model, when each client perturbs its data to guarantee privacy, the noise in the aggregated result is superfluous and may decrease the utility.
% Several solutions that combine differential privacy and secure multiparty computation (MPC) have been proposed to guarantee privacy and utility. The first key point to solve the problem is deploying MPC instead of the strong assumption of a trusted server. The second point is to reduce redundant noise by ensuring all clients perturbs the data collaboratively rather than independently.

The fundamental idea of differential privacy~\cite{dwork2006differential} is to perturb the query on a database such that the influence of each record in the database is bounded. The original definition of \differentialprivacy assumes the existence of a centralized and trusted server that manages the database and processes the queries. Subsequent work~\cite{dwork2006our} extends \differentialprivacy to a distributed setting, where the central server is replaced by several mutually distrustful and potentially malicious computation parties. To the best of our knowledge, Dwork et al.~\cite{dwork2006our} were the first to deploy malicious \smpc protocols to generate noise, compute the query with distributed data and perturb the query result. Generally, the works that attempt to realize \ddp can be categorized into two groups: the central \differentialprivacy model and the local \differentialprivacy model.
In the central \differentialprivacy model, a trusted central server responsible for computing the aggregate statistics and perturbing it is simulated by several semi-honest (or malicious) computation parties with \smpc. It is typically assumed that the majority of computation parties are not colluding. However, \smpc incurs high computation and communication overhead that reduces the efficiency and scalability of the model.
In the local \differentialprivacy model, the server is not trusted anymore. The users generate independent noise to privatize their data before sending it to the server. Therefore, the accuracy of the aggregate statistics is limited as the randomization is applied multiple times. Both models face a series of open challenges, and we discuss the details below.

\subsection{Local DP Model}
\label{subsec:LocallDPModel}
The existing solution to local \differentialprivacy model~\cite{rastogi2010differentially,Elaine2011Privacy,acs2011have, chan2012privacy,bindschaedler2017achieving,shi2017distributed,truex2019hybrid} relied on homomorphic encryption, \smpc and the infinite divisibility of certain probability distribution (e.g., the Laplace distribution~\cite{kotz2001laplace} or the Gaussian distribution). Specifically, each user perturbs their data independently and encrypts it under a homomorphic encryption scheme such that the server can aggregate the encrypted data and only reveal the noisy aggregated result.
Instead of using homomorphic encryption, other works~\cite{byrd2020differentially,balle2020private,ghazi2021differentially} had the users mask the locally perturbed data with additional noise and send the masked data to the server. After the aggregation of the server, the additional noise is canceled out, and only the noisy result for satisfying \differentialprivacy is revealed.
However, the existing works of the local \differentialprivacy model face two significant challenges. The first challenge is that the collusion users can subtract their noise term from the revealed result, weaken or break the \differentialprivacy guarantee. Therefore, in order to achieve the required \differentialprivacy guarantee, each user has to add a larger amount of noise, which leads a reduced utility of the aggregated result.
The second challenge is that the users have to pay a significant amount of computation effort, which makes the local \differentialprivacy model less practical for devices with limited computation power.


\subsection{Central DP Model}
\label{subsec:CentralDPModel}

For the central \differentialprivacy model, prior works~\cite{dwork2006our, eigner2014differentially,wu2016inherit,jayaraman2018distributed,knott2021crypten,yuan2021practical,eriguchi2021efficient} proposed a variety of methods to satisfy \differentialprivacy by generating distributed noise with \smpc.
Dwork et al.~\cite{dwork2006our} sampled noise from two distributions: Gaussian distribution (approximated with binomial distribution) and discrete Laplace distribution (approximated with Poisson distribution).
To satisfy $\left(\varepsilon ,\delta \right) $-\differentialprivacy with Gaussian noise, it needs to process $n\geq 64 \log_2{\left(2/\delta \right) }/\varepsilon ^2$ (e.g., $\varepsilon =0.01 \text{, }\delta=0.0001\Rightarrow n\approx 2^{23}$ ) uniform random bits in \smpc to generate binomial noise, that would lead to high \smpc overhead.
For discrete Laplace noise, the protocol requires to securely evaluate a circuit in \smpc to generate biased bits. However, the evaluation of the circuit fails with non-zero probability and needs multiple iterations to make the failure probability negligible.
Eigner et al.~\cite{eigner2014differentially} proposed an architecture called PrivaDA, that combined \differentialprivacy and \smpc, and generated Laplace noise and discrete Laplace noise in \smpc protocols. However, the generated Laplace noise suffers from the floating-point attack~\cite{mironov2012significance}. It remains to prove whether the discrete Laplace noise secure against the floating-point attack because its generation procedure is similar to the Laplace noise.
Wu et al.~\cite{wu2016inherit} described methods for generating Bernoulli, Laplace, and Gaussian noise in \smpc. The Laplace noise is generated based on the central limit theory~\cite[Example 10.3.2]{athreya2006measure}, i.e., the aggregation of $n$ Bernoulli random variable $Bern\left(0.5\right)$ approximates a normal random variable $\mathcal{N} \left(0,\frac{1}{4}\right)$ because $\sqrt{n}\left(\frac{\sum_{i = 1}^{n}  Bern\left(0.5\right) }{n}-\mu\right) \approx \mathcal{N} \left(0,\frac{1}{4}\right)   $ when $n \rightarrow \infty $. However, there is no discussion about how to choose $n$ without affecting the \differentialprivacy guarantee in their work.

Jayaraman et al.~\cite{jayaraman2018distributed}, Knott et al.~\cite{knott2021crypten}, and Yuan et al.~\cite{yuan2021practical} presented distributed learning approaches that combine \smpc and \differentialprivacy by generating distributed Laplace noise and Gaussian noise with \smpc protocols. The protocols for Laplace noise are similar to the work of Eigner et al.~\cite{eigner2014differentially}, and the protocols for Gaussian noise are all based on the Box-Muller sampling algorithm~\cite{box1958note}. However, Jin et al.~\cite{jin2022we} demonstrated a floating-point attack against the Box-Muller method.
Eriguchi et al.~\cite{eriguchi2021efficient} provided \smpc protocols to generate two types of noise: the \fdl noise and the binomial noise. In contrast to the discrete Laplace distribution that can sample arbitrarily large integers with very low probability, \fdl can only generate integers in a given range $\left[-N,N\right] $. The protocol for binomial noise deploys pseudorandom secret-sharing~\cite{cramer2005share} for generating shares of uniform random variables. However, the binomial noise only satisfies computational differential privacy~\cite{mironov2009computational}, that is a relaxation of the standard differential privacy definition~\cite{dwork2014algorithmic} and only secure against computationally bounded adversaries.
Our work provides an alternative solution to the central \differentialprivacy model by \textit{securely} generating distributed noise that is not affected by the attacks~\cite{mironov2012significance,jin2022we}.


% ??? combine floating-point with fixed-point
% \TODO{literature about floating-point and fixed-point MPC protocols}
\section{Arithmetic Operations in SMPC}
\label{ArithmeticOperationsinSMPC}
% This work is built upon the MOTION~\cite{braun2022motion} framework that supports Arithmetic Sharing, Boolean sharing with \gmw and Yao sharing with BMR.

Generally, most \smpc protocols that support arithmetic operations in \smpc are based on the binary circuit approach or the \lsss. In the binary circuit based approach, the arithmetic operations are represented as a Boolean circuit and evaluated with Yao's Garbled circuit protocol~\cite{yao1986generate} (\bmr~\cite{beaver1990round} for multi-party settings) or Boolean \gmw protocol~\cite{goldreich1987play}.
By contrast, in the \lsss-based approach~\cite{chaum1988multiparty,ben1988completeness}, the parties split their secret values into shares over a field $\mathbb{F} _q$ (or ring $\mathbb{Z} _{2^{\ell}}$) and send it to each of the parties. Next, we introduce the relevant works of arithmetic operations.

\subsection{Binary Circuit Based SMPC}
\label{subsec:BinaryCircuitBasedSMPC}

The binary circuits are typically composed of $\operatorname{XOR}$, $\operatorname{NOT}$, and $\operatorname{AND}$ gates, where the $\operatorname{AND}$ gates cause the primary cost. Therefore, the binary circuits should be optimized based on the cost metric of \smpc protocols. One method to generate low $\operatorname{AND}$-depth and low $\operatorname{AND}$-size binary circuits is to use the CBMC-GC~\cite{buscher2016compiling} circuit compiler that derives circuit design with the C program. Pullonen and Siim~\cite{pullonen2015combining} used CBMC-GC~\cite{buscher2016compiling} circuit compiler to generate size-optimized circuits for IEEE 754 floating-point operations with the C library~\cite{BerkelySoftFloat,musllibc}. Archer et al.~\cite{archer2021cost} applied CBMC-GC~\cite{buscher2016compiling} to generate depth-optimized circuits for IEEE 754 floating-point operations. In this work, we use CBMC-GC~\cite{archer2021cost} to generate depth-optimized circuits and size-optimized circuits for fixed-point and floating-point operations. We also use the available depth-optimized circuits for floating-point operations from work~\cite{demmler2015aby}.

\subsection{LSSS-Based SMPC}
\label{subsec:LSSS-BasedSMPC}
In order to guarantee the efficiency of the \smpc protocols, prior works~\cite{catrina2010secure,liedel2012secure,hemenway2016high,aly2019benchmarking,lu2020faster} deployed fixed-point rather than floating-point to represent real number operations.
Catrina and Saxena~\cite{catrina2010secure} built a series of fixed-point operations (e.g., addition, subtraction, multiplication, and division) by representing a fixed-point $x$ as:$x = \overline{x}\cdot  2^{-f}$, where $\overline{x}$ is an integer in a finite field $\mathbb{F} _q$, and $f$ is the length of the fraction bits.
The works~\cite{liedel2012secure,hemenway2016high,aly2019benchmarking,lu2020faster} proposed protocols for fixed-point operations such as exponential, square root, natural logarithm, and trigonometric functions with polynomial approximation~\cite{hart1978computer} or Goldschmidt approximation~\cite{markstein2004software}.

Another line of works~\cite{aliasgari2012secure,krips2014hybrid,kamm2015secure,rathee2022secfloat} focused on floating-point operations.
Aliasgari et al.~\cite{aliasgari2012secure} used a quadruple $\left(v, p, z, s\right) $ to represent a floating-point number $u$ as: $u= \left(1-2s\right) \cdot \left(1-z\right) \cdot v \cdot 2^p$. $v$, $p$, $z$, and $s$ are the mantissa, exponent, zero bit, and sign bit of $u$. Aliasgari et al.~\cite{aliasgari2012secure} also provided \smpc protocols for operations such as addition, subtraction, multiplication, divisibility, square root, logarithm, and exponentiation.
The subsequent works~\cite{krips2014hybrid,kamm2015secure,rathee2022secfloat} applied a similar form to represent the floating-point numbers.
Truex et al.~\cite{truex2019hybrid} proposed a hybrid method combing fixed-point and floating-point arithmetic, i.e., representing the mantissa of a floating-point number as a fixed-point number, and used \lsss-based fixed-point arithmetic when the mantissa is involved in the floating-point arithmetic. However, the fixed-point arithmetic is prone to overflow or underflow, and requires additional steps to correct the computation result that decreases the overall protocol performance.
Kamm and Willemson et al.~\cite{kamm2015secure} provided \smpc protocols for square root, natural exponentiation, and error function by approximating the functions with Taylor series expansion or Chebyshev polynomials.
% (ref. Optimizing MPC for Robust and Scalable Integer and Floating-Point Arithmetic) provide optimization for (hybrid model of ....) by parallelizing the polynomial approximation, eliminate branching by representing negative integer using two's complementation, etw.

Rathee et al.~\cite{rathee2022secfloat} built a precise and efficient 32-bit floating-point operation library (SecFloat) for \twopc. One highlight is the mixed-bitwidth computation technique, i.e., using low bitwidth to represent numbers as much as possible. The bitwidth conversion operations between different bitwidth are performed with specialized zero-extension and truncation \twopc protocols.
The second highlight is the use of low-degree polynomials to improve accuracy and efficiency.
One common method to compute functions such as natural exponentiation is polynomial approximations, where high-degree polynomials yield more accurate results but incur more computation and communication effort. Rathee et al.~\cite{rathee2022secfloat} replaced the high-degree polynomials with low-degree piecewise polynomials without affecting the accuracy. In specifically, for input $x\in \left(a,b\right) $, they approximated functions using different low-degree polynomials of $k$ subintervals ($\left(a, a_1\right) $, $\left(a_1, a_2\right)  $,$\ldots$, $\left(a_{k-1}, b\right) $). To determine the active interval of $x$, they deployed the \lut protocol~\cite{dessouky2017pushing} to compute the corresponding polynomial coefficients.
To explore if the efficiency improvement techniques of SecFloat~\cite{rathee2022secfloat} can be extended to multi-party settings, we implement the building blocks of SecFloat such as conversion operations between low-bitwidth and high-bitwidth, multi-party lookup table protocol~\cite{keller2017faster}, and \msnzb~\cite{rathee2021sirnn} in MITION~\cite{braun2022motion}.

Protocol $MSNZB\left(\left\langle x\right\rangle^A\right) $ computes the most significant non-zero bit index of arithmetic share $\left\langle x\right\rangle^A \in \mathbb{Z} _{2^{\ell}}$ and is a crucial building block of SecFloat.
We implement two types of \msnzb protocols in MOTION~\cite{braun2022motion} and compare their performance difference:
\begin{enumerate}
    \item $MSNZB\left(\left\langle x\right\rangle^A\right) $~\cite{aliasgari2012secure} uses operations such as bit-decomposition, Boolean sharing to arithmetic sharing conversion and arithmetic sharing addition to compute the most significant non-zero bit index of $x$.
    \item $MSNZB\left(\left\langle x\right\rangle^A\right) $~\cite{rathee2021sirnn} first decomposes the input $\left\langle x\right\rangle^A \in \mathbb{Z} _{2^{\ell}}$ into $\frac{\ell}{8}$ low-bitwidth arithmetic shares $\left\langle x_1\right\rangle^A$, $\left\langle x_2\right\rangle^A$, $\ldots$,  $\left\langle x_{\ell/8}\right\rangle^A\in \mathbb{Z} _{2^{{8}}} $. Then, each low-bitwidth arithmetic share is used as the input to the lookup table protocol~\cite{keller2017faster} to compute $MSNZB\left(\left\langle x_1\right\rangle^A\right) $, $\ldots$, $MSNZB\left(\left\langle x_{\ell/8}\right\rangle^A\right) $. Finally, $MSNZB\left(\left\langle x_1\right\rangle^A\right) $, $\ldots$, $MSNZB\left(\left\langle x_{\ell/8}\right\rangle^A\right) $ are combined together to compute $MSNZB\left(\left\langle x\right\rangle^A\right) $.
\end{enumerate}

\begin{table}[H]
    \caption{
        Online run-times in milliseconds (ms) for protocol \msnzb for the GMW (A). We take the average over 10 protocol runs in the LAN and WAN environments.
    }
    \label{tab:runtimes_secfloat}
    % \small
    \centering
    \nprounddigits{1} % remove fractional digits in this table
    \rowcolors{1}{gray!25}{white}
    % \resizebox{\columnwidth}{!}{
    \begin{tabular}{ l r r r r r r r r r r r}
        \toprule
        \hiderowcolors                          & \multicolumn{2}{c}{LAN} &        & \multicolumn{2}{c}{WAN}                         \\
        \cmidrule{2-3} \cmidrule{5-6} Operation &
        $N{=}3$                                 & $N{=}5$                 &        &
        $N{=}3$                                 & $N{=}5$                                                                            \\ \showrowcolors
        \midrule
        $MSNZB$~\cite{aliasgari2012secure}      & 18.41                   & 80.12  &                         & 886.06    & 1\,036.23 \\
        $MSNZB$~\cite{rathee2022secfloat}       & 359.76                  & 567.14 &                         & 5\,436.82 & 6\,355.90 \\
        \bottomrule
    \end{tabular}
    % }
\end{table}
\FloatBarrier

We can see in \autoref{tab:runtimes_secfloat} that the use of mixed-bitwidth and lookup table techniques does not bring efficiency improvement in the LAN (10Gbit/s Bandwidth, 1ms RTT) and WAN (100Mbit/s Bandwidth, 100ms RTT) networks. Details of benchmark environment are in~\autoref{cha:evaluation}. The reason is as follows:

% After benchmarking (cf.~\autoref{tab:runtimes_secfloat}), we found that the efficiency benefit brought by mixed-bitwidth becomes negligible when we extend it to multi-party settings.
\begin{enumerate}
    \item The conversion operations between low-bitwidth and high-bitwidth in SecFloat rely on a \twopc OT-based, highly efficient comparison protocol~\cite{rathee2020cryptflow2}, and it can not be directly extended to multi-party settings.
    \item In the two-party setting, when we take the value of two $\ell$-bit arithmetic shares $\left\langle a\right\rangle^A_0 $, $\left\langle a\right\rangle^A_1$ as plaintext values and compute the addition result in real number field, we need an $\left(\ell+1\right) $-bit integer $a =\left\langle a\right\rangle^A_0 +\left\langle a\right\rangle^A_1 $ to hold the addition result without overflow. The value of the most significant bit of $a$ is used during the bitwidth conversion operation of SecFloat. For $N\geq 3$ parties, the addition result of $N$ $\ell$-bit arithmetic plaintext values needs a $\left(\left\lceil \log_2{N}\right\rceil +l\right) $-bit integer to hold. The $\left\lceil \log_2{N}\right\rceil$ most significant bits are used for the bitwidth conversion operation in the multi-party setting. Hence, as the number of parties grows, the complexity of conversion operations also increases.
\end{enumerate}
Therefore, we build the \lsss-based floating-point operations in arithmetic sharing protocols with uniform bitwidth such as the work of Aliasgari et al.~\cite{aliasgari2012secure}.


% However, we find that SecFloat can't be extended to $N$-party computations ($N\geq 3$) while preserving its efficiency. First, SecFloat relies heavily on the Oblivious Transfer techniques (ref. crytpflow) for mix-bitwidth and $LUT$ operationsm which is not available in multi-party computations. To verify this, we implement the $MSNZB$ (most significant non-zero bit index) protocol deployed in SecFloat using the idea of mix-bitwidth and $LUT$ (ref. fast LUT) in MOTION framework, and compare it with a another implementation of $MSNZB$ that use Boolean GMW operation and share conversion.



% Pullonen and Siim~\cite{pullonen2015combining} presented another hybrid protocol that used Yao's Garbled Circuit protocol~\cite{yao1986generate} for bit-level operations and \lsss-based


% \textbf{MPC Protocols for Floating-Point Arithmetic}
% There are prior works that focus on floating-point arithmetic MPC protocols.
% (ref. Secure Floating-Point Arithmetic and Private Satellite Collision Analysis) provides AGMW based floating-point MPC protocols and using polynomial approximation for $exp$, $sqrt$, etw.


% (ref. Combining Secret Sharing and Garbled Circuits for Efficient Private IEEE 754 Floating-Point Computations) provides a hybrid protocol for 2-party floating-point arithmetic that convert bgmw sharing to Yao's sharing and evaluate arithmetic operation as Garbled circuits protocols.





% After the benchmarking, we found that for MPC frameworks that support multiparty computations, the efficiency benefit brought by mix-bitwidth would be negligible as $N$ increases. ??? zero-extension, LUT multiparty complexity analysis.



% (ref. The The Cost of IEEE Arithmetic in Secure Computation) implement LSSS-based and binary circuit-based floating-point arithmetic MPC protocols and compare their performance. In their benchmarking result, the LSSS-based floating-point operation is about $10-100x$ faster than binary circuit-based floating-point operation. However, in our implementation, the binary circuit-based floating-point operation is $5-10x$ faster than LSSS-based floating-based without SIMD and can be up to $1000x$ faster than LSSS-based floating-point operation when amortized over SIMD=1000.

% \textbf{MPC Protocols for Fixed-Point Arithmetic}
% (ref. High-precision Secure Computation of Satellite Collision Probabilities) provides methods for 2-party fixed-point arithmetic by combine AGMW and BGMW, i.e., using AGMW for integer addition and multiplication using AGMW and BGMW for integer comparison, shifting, $exp$, etw.

% (ref.  Benchmarking Privacy Preserving Scientific Operations) provides AGMW based fixed-point arithmetic.

% (ref. Round-Efficient Protocols for Secure Multiparty Fixed-Point Arithmetic) agmw fixed-point.






