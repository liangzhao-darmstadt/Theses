\chapter{Benchmark}
\label{cha:benchmark}


% % example plot
% \begin{figure}[H]
%     \centering
%     \begin{tikzpicture}[]
%         \begin{axis}[
%             legend style={at={(1.05, 0.5)}, anchor = west}, %place legend outside of diagram. right side, centered vertically.
%             xlabel={Number of Things},
%             ylabel={Count [Unit]},
%             %axis y discontinuity = parallel,
%             axis y line = left,
%             axis x line = bottom,
%             ymin = 0,
%             ymax = 4,
%             xmin = 0,
%             xmax = 9,
%             grid = both,
%             minor tick num = 1,
%             width = 7cm,
%             height = 5cm,
%             %legend pos = outer north east,
%             legend entries = {A, B},
%             legend cell align = left,
%         ]
%         % \addplot+ table[x=inputs, y=timeA] {data/example.dat};
%         % \addplot+ table[x=inputs, y=timeB] {data/example.dat};
%         \addplot+ table[x=inputs, y=timeA] {data/benchmark_liangzhao_arithmetic_gmw_operation_P0.csv};
%         \addplot+ table[x=inputs, y=timeB] {data/benchmark_liangzhao_arithmetic_gmw_operation_P0.csv};
%         \end{axis}
%     \end{tikzpicture}
%     \caption[Example PGF Plot]{Example Plot made with pgfplots. Here we visualize the data from~\autoref{tab:runtimes_aes_sha}.}
%     \label{fig:example_plot}
%     \end{figure}

\section{Arithmetic Gmw Operation}
\begin{table}[H]
	\centering
	\caption[Run-times in microseconds for arithmetic gmw operations (128-bit) of two parties]{Run-times in microseconds for arithmetic gmw operations (128-bit) of two parties}
	\label{tab:table_ex}
\pgfplotstabletypeset[
    multicolumn names, % allows to have multicolumn names
	col sep=comma,
	columns/Protocol/.style={string type}
]{data/benchmark_liangzhao_arithmetic_gmw_operation_P0.csv}
\end{table} 
\FloatBarrier

\section{Floating-Point Operations}
\begin{table}[H]
	\centering
	\caption[Run-times in microseconds for floating-point operations based on boolean circuit (bgmw, 64-bit, SIMD = 1) of two parties]{Run-times in microseconds for floating-point operations based on boolean circuit (bgmw, 64-bit, SIMD = 1) of two parties}
	\label{tab:table_ex}
\pgfplotstabletypeset[
    multicolumn names, % allows to have multicolumn names
	col sep=comma,
	columns/Protocol/.style={string type}
]{data/benchmark_liangzhao_floating_point_operation_bgmw_simd_1.csv}
\end{table} 
\FloatBarrier

\begin{table}[H]
	\centering
	\caption[Run-times in microseconds for floating-point operations based on boolean circuit amortized over 1000 SIMD (bgmw, 64-bit) of two parties]{Run-times in microseconds for floating-point operations based on boolean circuit amortized over 1000 SIMD (bgmw, 64-bit) of two parties}
	\label{tab:table_ex}
\pgfplotstabletypeset[
    multicolumn names, % allows to have multicolumn names
	col sep=comma,
	columns/Protocol/.style={string type}
]{data/benchmark_liangzhao_floating_point_operation_bgmw_simd_1000.csv}
\end{table} 
\FloatBarrier


\begin{table}[H]
	\centering
	\caption[Run-times in microseconds for floating-point operations based on arithmetic gmw (128-bit) of two parties]{Run-times in microseconds for floating-point operations based on arithmetic gmw (128-bit) of two parties}
	\label{tab:table_ex}
\pgfplotstabletypeset[
    multicolumn names, % allows to have multicolumn names
	col sep=comma,
	columns/Protocol/.style={string type}
]{data/benchmark_liangzhao_floating_point_operation_agmw.csv}
\end{table} 
\FloatBarrier

\section{Fixed-Point Operations}
\begin{table}[H]
	\centering
	\caption[Run-times in microseconds for fixed-point operations based on boolean circuit (bgmw, 64-bit, SIMD = 1) of two parties]{Run-times in microseconds for fixed-point operations based on boolean circuit (bgmw, 64-bit, SIMD = 1) of two parties}
	\label{tab:table_ex}
\pgfplotstabletypeset[
    multicolumn names, % allows to have multicolumn names
	col sep=comma,
	columns/Protocol/.style={string type}
]{data/benchmark_liangzhao_fixed_point_operation_bgmw_simd_1.csv}
\end{table} 
\FloatBarrier

\begin{table}[H]
	\centering
	\caption[Run-times in microseconds for fixed-point operations based on boolean circuit amortized over 1000 SIMD (bgmw, 64-bit) of two parties]{Run-times in microseconds for fixed-point operations based on boolean circuit amortized over 1000 SIMD (bgmw, 64-bit) of two parties}
	\label{tab:table_ex}
\pgfplotstabletypeset[
    multicolumn names, % allows to have multicolumn names
	col sep=comma,
	columns/Protocol/.style={string type}
]{data/benchmark_liangzhao_fixed_point_operation_bgmw_simd_1000.csv}
\end{table} 
\FloatBarrier


\begin{table}[H]
	\centering
	\caption[Run-times in microseconds for fixed-point operations based on arithmetic gmw (128-bit) of two parties]{Run-times in microseconds for fixed-point operations based on arithmetic gmw (128-bit) of two parties}
	\label{tab:table_ex}
\pgfplotstabletypeset[
    multicolumn names, % allows to have multicolumn names
	col sep=comma,
	columns/Protocol/.style={string type}
]{data/benchmark_liangzhao_fixed_point_operation_agmw.csv}
\end{table} 
\FloatBarrier

\section{DP Mechanism}
\begin{table}[H]
	\centering
	\caption[Run-times in microseconds for DP mechanism based on boolean circuit (bgmw, 64-bit, SIMD = 1) of two parties]{Run-times in microseconds for DP mechanism based on boolean circuit (bgmw, 64-bit, SIMD = 1) of two parties}
	\label{tab:table_ex}
\pgfplotstabletypeset[
    multicolumn names, % allows to have multicolumn names
	col sep=comma,
	columns/Protocol/.style={string type}
]{data/benchmark_liangzhao_dp_mechanism_simd_1.csv}
\end{table} 
\FloatBarrier

\begin{table}[H]
	\centering
	\caption[Run-times in microseconds for DP mechanism based on boolean circuit amortized over 1000 SIMD (bgmw, 64-bit) of two parties]{Run-times in microseconds for DP mechanism based on boolean circuit amortized over 1000 SIMD (bgmw, 64-bit) of two parties}
	\label{tab:table_ex}
\pgfplotstabletypeset[
    multicolumn names, % allows to have multicolumn names
	col sep=comma,
	columns/Protocol/.style={string type}
]{data/benchmark_liangzhao_dp_mechanism_simd_1000.csv}
\end{table} 
\FloatBarrier






