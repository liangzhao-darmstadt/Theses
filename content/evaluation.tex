\chapter{Evaluation}
\label{cha:evaluation}

% Recall that the two challenges that we proposed in~\autoref{cha:MPCProtocolsforDifferentiallyPrivateMechanisms}:
% \begin{enumerate}
%     \item What are the most efficient MPC protocols for arithmetic operations?
%     \item What are the most efficient sampling algorithms for differentially private mechanisms in MPC?
% \end{enumerate}

% To answer the above questions, 

In this chapter, we implement and measure the performance of all the protocols discussed in~\autoref{cha:MPCProtocolsforDifferentiallyPrivateMechanisms} in a semi-honest scenario.
First, we benchmark the performance of fixed-point and floating-point operations in different \smpc protocols (\booleanGMW and \arithmeticGMW) and choose the most efficient \smpc protocols to build the differentially private mechanisms. For comparison, we also implement the \textit{insecure} differentially private mechanisms~\cite{eigner2014differentially} in MOTION~\cite{braun2022motion}.
% Then, we implement the differentially private mechanisms and corresponding sampling algorithms.
Our code is public on GitHub~\footnote{https://github.com/liangzhao-darmstadt/MOTION};


% The protocol that we implemented support $N \geq 3$ parties and is secure against up to $N-1$ passive corruption parties. 
% We compare our MPC-DP protocols with other work (ref. ) by implementing their protocols in MOTION.

\textbf{Experimental Setup.}
The experiments are performed in five connected servers (Intel Core i9-7960X process, 128GB RAM, 10 Gbps network). We define two network settings to analyze the performance of our \smpc protocols.
\begin{enumerate}
    \item LAN10: 10Gbit/s Bandwidth, 1ms RTT.
    \item WAN: 100Mbit/s Bandwidth, 100ms RTT.
\end{enumerate}

\section{Arithmetic Operations Benchmarks}
\label{sec:ArithmeticOperationsPerformanceEvaluation}
In this section, we compare the performance of arithmetic operations in different data types (\booleanGMW fixed/floating-point, \arithmeticGMW fixed/floating-point).
All the arithmetic operations and underlying sub-protocols were tested for correctness with Google Test~\footnote{https://github.com/google/googletest}.
Specifically, the following arithmetic operations were benchmarked: addition, subtraction, multiplication, division, exp2, log2, square root, comparison ($<$), and data type conversions.
We measure the total runtime (offline + online) in microseconds (ms) and average the runtime results over $10$ iterations.

\paragraph{Fixed-Point Benchmarks.}
\label{para:Fixed-PointBenchmarking}
% \TODO{add benchmark result of BMR protocols, BMR protocol is only 1x-1.5x faster than BGMW in division and slower in all other operations. Therefore, we choose BGMW as main protocols}
In the experiments, we use the \booleanGMW fixed-point numbers with total bit-length k=$64$ and bit-length f=$16$ for the fractional part.
For the \arithmeticGMW fixed-point numbers, we use k=$41$ for the total bit-length and f=$20$ for the bit-length of the fractional part.
We compare the \booleanGMW and \arithmeticGMW fixed-point arithmetic operations in \autoref{fig:fixedpointBenchmarks}. As \autoref{fig:fixedpointBenchmarks} suggests, addition and subtraction are very fast for \arithmeticGMW fixed-point because they could be computed locally with the precomputed \mts (cf.~\autoref{subsubsec:MTs}). For the multiplication operation, the \booleanGMW and \arithmeticGMW fixed-point have very close performance, while the arithemtic truncation operation~\cite{catrina2010secure} is the major overhead of \arithmeticGMW fixed-point. The division operation is an expensive operation for both \booleanGMW and \arithmeticGMW fixed-point, where the \arithmeticGMW fixed-point is $1.21\times-2.67\times$ faster than the \booleanGMW fixed-point. The base-2 exponential function (exp2) and the binary logarithm (log2) are substantially faster for the \booleanGMW than the \arithmeticGMW fixed-point (exp2: $7.64\times-11.98\times$, log2: $5.89\times-7.43\times$). For the square root operation, the \booleanGMW fixed-point and \arithmeticGMW fixed-point have close performance in the \twopc and \threepc settings, whereas \booleanGMW fixed-point is $2.06\times$ faster than the \arithmeticGMW fixed-point in the five-party settings. The comparision operation between \booleanGMW and \arithmeticGMW fixed-point are very close to each other, while the \booleanGMW fixed-point is more efficient when the number of computation parties is greater than three. The conversion operation from fixed-point to integer (fx2int) is up to $22.93\times$ faster for the \booleanGMW than the \arithmeticGMW fixed-point. The remaining conversion operation (fx2fl) is up to $2.10\times$ faster for the \arithmeticGMW than the \booleanGMW fixed-point.
Finally, we choose \booleanGMW fixed-point as the primary data type for fixed-point operations because of the advantages in operations such as exp2 and fx2int.


\paragraph{Floating-Point Benchmarks.}
\label{para:Floating-PointBenchmarking}
We use the \booleanGMW floating-point numbers that have a $53$-bit mantissa, an $11$-bit exponent, and an $1$-bit sign. In contrast, the \arithmeticGMW floating-point numbers are represented by a quadruple $\left(v, p, z, s\right) $ (cf.~\autoref{para:AGMWFloating-Point}), where $v$, $p$, $z$, and $s$ are represented by $128$-bit arithmetic shares. As \autoref{fig:floatingpointBenchmarks} shows, the \booleanGMW floating-point is faster than the \arithmeticGMW floating-point in almost all the arithmetic operations except for the division operation in \twopc settings. For addition, subtraction and multiplication operations, the \booleanGMW floating-point is $1.59\times-4.22\times$ faster than the \arithmeticGMW floating-point. The primary overhead of \arithmeticGMW is the truncation and bit-decomposition operations~\cite{aliasgari2012secure}. The division operation is slightly faster for the \arithmeticGMW than the \booleanGMW floating-point only in the five-party setting. For base-2 exponential function (exp2), binary logarithm (log2) and square root operations, the \booleanGMW floating-point is substantially faster than the \arithmeticGMW floating-point by a factor up to $13.94\times$. The reason for the low efficient performance of the \arithmeticGMW floating-point is the multiply invocations of the floating-point multiplication as subroutine in these operations~\cite{aliasgari2012secure}. The comparison operation of \arithmeticGMW floating-point is more expensive than that in the \booleanGMW floating-point. For the remaining conversion operations (fl2int and fl2fx), the \booleanGMW floating-point is more efficient as the conversion operation to integer or fixed-point can be realized through bit manipulation at low cost. Based on the above analysis, we choose \booleanGMW floating-point for floating-point arithmetic operations.



\begin{figure}[ht]
    \centering
    % scatter plot of member decompose-types-bdd-graph-weak member types
    \begin{tikzpicture}[scale=1,transform shape]
        \begin{groupplot}[
                group style={
                        group name=my plots,
                        group size=2 by 5,
                        xlabels at=edge bottom,
                        ylabels at=edge left,
                        horizontal sep=2cm,
                        vertical sep=1.5cm,
                    },
                width=0.5\linewidth
            ]
            %% label = 
            \nextgroupplot[    %place legend outside of diagram. right side, centered vertically.
                % xlabel={Number of Parties},
                % ylabel={Online Runtime (ms)},
                %axis y discontinuity = parallel,
                axis y line = left,
                axis x line = bottom,
                ymin = 0,
                ymax = 200,
                xmin = 1.5,
                xmax = 5.5,
                grid = both,
                minor tick num = 1,
                width = 7cm,
                height = 3.8cm,
                title=\textbf{(a) addition},
                anchor=north,
                %legend pos = outer north east,
                % legend entries = {Fixed-Point BGMW, Fixed-Point AGMW},
                % legend cell align = left,
            ]

            \addplot+ table[x=parties, y=Fx_circuit] {data/fixed_point_add.dat};
            \addplot+ table[x=parties, y=Fx_agmw] {data/fixed_point_add.dat};

            \nextgroupplot[    %place legend outside of diagram. right side, centered vertically.
                % xlabel={Number of Parties},
                % ylabel={Online Runtime (ms)},
                %axis y discontinuity = parallel,
                axis y line = left,
                axis x line = bottom,
                ymin = 0,
                ymax = 200,
                xmin = 1.5,
                xmax = 5.5,
                grid = both,
                minor tick num = 1,
                width = 7cm,
                height = 3.8cm,
                title=\textbf{(b) subtraction},
                %legend pos = outer north east,
                % legend entries = {Fixed-Point BGMW, Fixed-Point AGMW},
                % legend cell align = left,
            ]

            \addplot+ table[x=parties, y=Fx_circuit] {data/fixed_point_sub.dat};
            \addplot+ table[x=parties, y=Fx_agmw] {data/fixed_point_sub.dat};


            \nextgroupplot[    %place legend outside of diagram. right side, centered vertically.
                % xlabel={Number of Parties},
                % ylabel={Online Runtime (ms)},
                %axis y discontinuity = parallel,
                axis y line = left,
                axis x line = bottom,
                ymin = 0,
                ymax = 450,
                xmin = 1.5,
                xmax = 5.5,
                grid = both,
                minor tick num = 1,
                width = 7cm,
                height = 3.8cm,
                title=\textbf{(c) mulitplication},
                %legend pos = outer north east,
                % legend entries = {Fixed-Point BGMW, Fixed-Point AGMW},
                % legend cell align = left,
            ]

            \addplot+ table[x=parties, y=Fx_circuit] {data/fixed_point_mul.dat};
            \addplot+ table[x=parties, y=Fx_agmw] {data/fixed_point_mul.dat};
            \nextgroupplot[    %place legend outside of diagram. right side, centered vertically.
                % xlabel={Number of Parties},
                % ylabel={Online Runtime (ms)},
                %axis y discontinuity = parallel,
                axis y line = left,
                axis x line = bottom,
                ymin = 0,
                ymax = 4000,
                xmin = 1.5,
                xmax = 5.5,
                grid = both,
                minor tick num = 1,
                width = 7cm,
                height = 3.8cm,
                title=\textbf{(d) division},
                %legend pos = outer north east,
                % legend entries = {Fixed-Point BGMW, Fixed-Point AGMW},
                % legend cell align = left,
            ]

            \addplot+ table[x=parties, y=Fx_circuit] {data/fixed_point_div.dat};
            \addplot+ table[x=parties, y=Fx_agmw] {data/fixed_point_div.dat};
            \nextgroupplot[    %place legend outside of diagram. right side, centered vertically.
                % xlabel={Number of Parties},
                % ylabel={Online Runtime (ms)},
                %axis y discontinuity = parallel,
                axis y line = left,
                axis x line = bottom,
                ymin = 0,
                ymax =6000,
                xmin = 1.5,
                xmax = 5.5,
                grid = both,
                minor tick num = 1,
                width = 7cm,
                height = 3.8cm,
                title=\textbf{(e) exp2},
                %legend pos = outer north east,
                % legend entries = {Fixed-Point BGMW, Fixed-Point AGMW},
                % legend cell align = left,
            ]

            \addplot+ table[x=parties, y=Fx_circuit] {data/fixed_point_exp2.dat};
            \addplot+ table[x=parties, y=Fx_agmw] {data/fixed_point_exp2.dat};
            \nextgroupplot[    %place legend outside of diagram. right side, centered vertically.
                % xlabel={Number of Parties},
                % ylabel={Online Runtime (ms)},
                %axis y discontinuity = parallel,
                axis y line = left,
                axis x line = bottom,
                ymin = 0,
                ymax = 6000,
                xmin = 1.5,
                xmax = 5.5,
                grid = both,
                minor tick num = 1,
                width = 7cm,
                height = 3.8cm,
                title=\textbf{(f) log2},
                %legend pos = outer north east,
                % legend entries = {Fixed-Point BGMW, Fixed-Point AGMW},
                % legend cell align = left,
            ]

            \addplot+ table[x=parties, y=Fx_circuit] {data/fixed_point_log2.dat};
            \addplot+ table[x=parties, y=Fx_agmw] {data/fixed_point_log2.dat};
            \nextgroupplot[    %place legend outside of diagram. right side, centered vertically.
                % xlabel={Number of Parties},
                % ylabel={Online Runtime (ms)},
                %axis y discontinuity = parallel,
                axis y line = left,
                axis x line = bottom,
                ymin = 0,
                ymax = 3500,
                xmin = 1.5,
                xmax = 5.5,
                grid = both,
                minor tick num = 1,
                width = 7cm,
                height = 3.8cm,
                title=\textbf{(g) square root},
                %legend pos = outer north east,
                % legend entries = {Fixed-Point BGMW, Fixed-Point AGMW},
                % legend cell align = left,
            ]

            \addplot+ table[x=parties, y=Fx_circuit] {data/fixed_point_sqrt.dat};
            \addplot+ table[x=parties, y=Fx_agmw] {data/fixed_point_sqrt.dat};
            \nextgroupplot[    %place legend outside of diagram. right side, centered vertically.
                % xlabel={Number of Parties},
                % ylabel={Online Runtime (ms)},
                %axis y discontinuity = parallel,
                axis y line = left,
                axis x line = bottom,
                ymin = 0,
                ymax = 400,
                xmin = 1.5,
                xmax = 5.5,
                grid = both,
                minor tick num = 1,
                width = 7cm,
                height = 3.8cm,
                title=\textbf{(h) comparison},
                %legend pos = outer north east,
                % legend entries = {Fixed-Point BGMW, Fixed-Point AGMW},
                % legend cell align = left,
            ]

            \addplot+ table[x=parties, y=Fx_circuit] {data/fixed_point_lt.dat};
            \addplot+ table[x=parties, y=Fx_agmw] {data/fixed_point_lt.dat};
            \nextgroupplot[    %place legend outside of diagram. right side, centered vertically.
                % xlabel={Number of Parties},
                % ylabel={Online Runtime (ms)},
                %axis y discontinuity = parallel,
                axis y line = left,
                axis x line = bottom,
                ymin = 0,
                ymax = 3500,
                xmin = 1.5,
                xmax = 5.5,
                grid = both,
                minor tick num = 1,
                width = 7cm,
                height = 3.8cm,
                title=\textbf{(i) fx2int},
                %legend pos = outer north east,
                % legend entries = {Fixed-Point BGMW, Fixed-Point AGMW},
                % legend cell align = left,
            ]

            \addplot+ table[x=parties, y=Fx_circuit] {data/fixed_point_fx2int.dat};
            \addplot+ table[x=parties, y=Fx_agmw] {data/fixed_point_fx2int.dat};

            \nextgroupplot[
                % label style={font=\tiny}, %place legend outside of diagram. right side, centered vertically.
                % xlabel={Number of Parties},
                % ylabel={Online Runtime (ms)},
                %axis y discontinuity = parallel,
                axis y line = left,
                axis x line = bottom,
                ymin = 0,
                ymax = 1500,
                xmin = 1.5,
                xmax = 5.5,
                grid = both,
                minor tick num = 1,
                width = 7cm,
                height = 3.8cm,
                title=\textbf{(j) fx2fl},
                %legend pos = outer north east,
                % legend entries = {Fixed-Point BGMW, Fixed-Point AGMW},
                % legend cell align = left,
                legend style={at={(0.5,-0.2)},anchor=north},
                % label style={font=\tiny},
                legend to name=UniversalLegend
            ]

            \addplot+ table[x=parties, y=Fx_circuit] {data/fixed_point_fx2fl.dat};
            \addplot+ table[x=parties, y=Fx_agmw] {data/fixed_point_fx2fl.dat};

            \legend{\booleanGMW Fixed-Point  w/o \simd, \arithmeticGMW Fixed-Point }
        \end{groupplot}
    \end{tikzpicture}\\
    \ref{UniversalLegend}
    \caption[Total runtime in microseconds for the \booleanGMW and \arithmeticGMW fixed-point arithmetic operations.]{Total runtime in microseconds (y-axis) for the \booleanGMW and \arithmeticGMW fixed-point arithmetic operations in WAN test encironments with different parties (x-axis). We take the average results over $10$ protocol runs. }
    \label{fig:fixedpointBenchmarks}
\end{figure}
\FloatBarrier








% \autoref{tab:runtimes_fixed_point_circuit} shows that division is the slowest operation for BGMW fixed-point arithmetic. The reason is that the circuit we generated using HyCC~\cite{buscher2018hycc} have a deeper depth of $6\,432$. In the LAN setting, the $5$-party division is $2.2\times $ slower than the $3$-party division, whereas, in the WAN setting, the factor was $1.1$.

% In \autoref{tab:runtimes_fixed_point_agmw}, natural exponentiation has the longest online run-times. In the LAN setting, as the number of parties increases from $3$ to $5$, the online run-times of operation subtraction and square root increase at most by a factor $15\times$ and $3.8\times$.

% Comparing the online run-times of fixed-point arithmetic from \autoref{tab:runtimes_fixed_point_circuit} and \autoref{tab:runtimes_fixed_point_agmw}, we can see that the BGMW-based fixed-point arithmetic (with $SIMD=1$) is more efficient than the AGMW-based fixed-point arithmetic only in operations such as exp2, exp and conversion operation with integer, but slower in all other operations.
% However, when we apply the SIMD technique, BGMW-based fixed-point arithmetic (with $SIMD=1000$) is $4\times-4000\times$ faster than the AGMW-based fixed-point arithmetic for all the operations in the LAN setting. In the WAN setting, the BGMW-based fixed-point arithmetic is $35\times-2800\times$ faster than the AGMW-based fixed-point arithmetic except for addition and multiplication operations.
% That means if the functionalities (e.g., \textbf{FOR} loop in $\Uppi^{prot:TwoSideGeometric}$) can be parallelized into independent and identical operations, the BGMW-based fixed-point arithmetic is a better option than AGMW-based fixed-point arithmetic in the LAN setting.
% Therefore, we choose BGMW-based fixed-point arithmetic to implement the MPC protocols.


% \begin{table}
%     \caption{
%         Online run-times in milliseconds (ms) of fixed-point (k=64, f=16) operations for the GMW (B) and \TODO{add BMR protocol (Y)}.
%         For the entries with $SIMD=1\,000$, we specify the run-time of a single operation amortized over $ 1\,000 $ SIMD values.
%         We take the average over 10 protocol runs in the LAN and WAN environments.
%     }
%     \label{tab:runtimes_fixed_point_circuit}
%     % \small
%     \centering
%     \nprounddigits{1} % remove fractional digits in this table
%     \rowcolors{1}{gray!25}{white}
%     \resizebox{\columnwidth}{!}{
%         \begin{tabular}{ l c c r r r r r r r r r r}
%             \toprule
%             \hiderowcolors                             &         & \multicolumn{3}{c}{LAN} &            & \multicolumn{3}{c}{WAN}                                              \\
%             \cmidrule{3-5} \cmidrule{7-8}  Parties $N$ & SIMD    &
%             $N{=}2$                                    & $N{=}3$ & $N{=}5$                 &            &
%             $N{=}2$                                    & $N{=}3$ & $N{=}5$                                                                                                     \\ \showrowcolors
%             \midrule
%             $FX\_Add^{\BOOL}$                          & 1       & 27.49                   & 29.22      & 39.75                   &  & 507.80      & 516.25      & 556.10      \\
%             $FX\_Sub^{\BOOL}$                          & 1       & 23.74                   & 60.90      & 39.36                   &  & 527.04      & 565.47      & 562.13      \\
%             $FX\_Mul^{\BOOL}$                          & 1       & 197.14                  & 199.17     & 219.78                  &  & 1\,487.11   & 1\,772.54   & 1\,829.94   \\
%             $FX\_Div^{\BOOL}$                          & 1       & 25\,690.22              & 26\,924.64 & 59\,848.43              &  & 336\,749.59 & 354\,113.67 & 391\,287.07 \\
%             $FX\_Lt^{\BOOL}$                           & 1       & 28.63                   & 96.86      & 183.17                  &  & 885.07      & 867.33      & 995.09      \\
%             $FX\_Exp2^{\BOOL}$                         & 1       & 239.48                  & 259.27     & 255.51                  &  & 4\,924.61   & 5\,330.73   & 5\,566.90   \\
%             $FX\_Log2^{\BOOL}$                         & 1       & 2\,880.34               & 2\,898.00  & 2\,887.24               &  & 17\,599.92  & 19\,747.61  & 22\,030.32  \\
%             $FX\_Exp^{\BOOL}$                          & 1       & 297.63                  & 283.95     & 256.11                  &  & 5\,482.55   & 5\,971.34   & 6\,326.46   \\
%             $FX\_Ln^{\BOOL}$                           & 1       & 2\,910.26               & 2\,995.85  & 2\,985.79               &  & 18\,903.29  & 19\,957.76  & 22\,555.63  \\
%             $FX\_Sqrt^{\BOOL}$                         & 1       & 1\,277.98               & 1\,749.13  & 2\,022.90               &  & 19\,957.41  & 20\,955.41  & 23\,665.47  \\
%             $Fx2Int^{\BOOL}$                           & 1       & 14.35                   & 15.53      & 17.53                   &  & 675.69      & 795.63      & 948.14      \\
%             $Fx2FL^{\BOOL}$                            & 1       & 683.13                  & 722.25     & 1\,271.52               &  & 10\,781.96  & 11\,794.16  & 12\,818.66  \\
%             \midrule
%             $FX\_Add^{\BOOL}$                          & 1\,000  & 0.03                    & 0.04       & 0.04                    &  & 0.54        & 0.48        & 0.79        \\
%             $FX\_Sub^{\BOOL}$                          & 1\,000  & 0.01                    & 0.01       & 0.04                    &  & 0.51        & 0.56        & 0.52        \\
%             $FX\_Mul^{\BOOL}$                          & 1\,000  & 0.10                    & 0.13       & 0.17                    &  & 1.12        & 2.47        & 2.24        \\
%             $FX\_Div^{\BOOL}$                          & 1\,000  & 24.71                   & 25.25      & 58.90                   &  & 337.73      & 357.79      & 391.94      \\
%             $FX\_Lt^{\BOOL}$                           & 1\,000  & 0.05                    & 0.15       & 0.14                    &  & 0.79        & 0.80        & 1.00        \\
%             $FX\_Exp2^{\BOOL}$                         & 1\,000  & 0.21                    & 0.41       & 0.47                    &  & 4.75        & 5.68        & 6.14        \\
%             $FX\_Log2^{\BOOL}$                         & 1\,000  & 2.61                    & 1.80       & 1.57                    &  & 16.35       & 20.13       & 26.91       \\
%             $FX\_Exp^{\BOOL}$                          & 1\,000  & 0.24                    & 0.45       & 0.55                    &  & 5.66        & 6.37        & 6.65        \\
%             $FX\_Ln^{\BOOL}$                           & 1\,000  & 4.35                    & 3.93       & 1.70                    &  & 17.50       & 20.41       & 33.14       \\
%             $FX\_Sqrt^{\BOOL}$                         & 1\,000  & 4.62                    & 4.33       & 5.11                    &  & 19.44       & 21.60       & 32.91       \\
%             $Fx2Int^{\BOOL}$                           & 1\,000  & 0.01                    & 0.02       & 0.05                    &  & 0.62        & 0.56        & 0.80        \\
%             $Fx2FL^{\BOOL}$                            & 1\,000  & 0.25                    & 0.40       & 0.55                    &  & 7.36        & 9.64        & 10.38       \\
%             \bottomrule
%         \end{tabular}
%     }
% \end{table}
% \FloatBarrier


% \begin{table}
%     \caption{
%         Online run-times in milliseconds (ms) of fixed-point (k=41, f=20) operations for the GMW (A). We take the average over 10 protocol runs in the LAN and WAN environments.
%     }
%     \label{tab:runtimes_fixed_point_agmw}
%     % \small
%     \centering
%     \nprounddigits{1} % remove fractional digits in this table
%     \rowcolors{1}{gray!25}{white}
%     \resizebox{\columnwidth}{!}{
%         \begin{tabular}{ l c r r r r r r r r r r}
%             \toprule
%             \hiderowcolors                             & \multicolumn{3}{c}{LAN} &           & \multicolumn{3}{c}{WAN}                                            \\
%             \cmidrule{2-4} \cmidrule{6-8}  Parties $N$ &
%             $N{=}2$                                    & $N{=}3$                 & $N{=}5$   &                         &
%             $N{=}2$                                    & $N{=}3$                 & $N{=}5$                                                                        \\ \showrowcolors
%             \midrule
%             $FX\_Add^{\ARITH}$                         & 0.10                    & 0.23      & 0.24                    &   & 0.19       & 0.25       & 0.28       \\
%             $FX\_Sub^{\ARITH}$                         & 0.14                    & 0.22      & 3.45                    &   & 0.27       & 0.24       & 0.46       \\
%             $FX\_Mul^{\ARITH}$                         & 16.05                   & 38.24     & 25.20                   &   & 531.76     & 437.77     & 448.93     \\
%             $FX\_Div^{\ARITH}$                         & 219.16                  & 755.79    & 1\,650.56               &   & 8\,968.31  & 10\,198.43 & 11\,210.29 \\
%             $FX\_Lt^{\ARITH}$                          & 19.32                   & 13.00     & 18.26                   &   & 350.21     & 334.68     & 372.35     \\
%             $FX\_Exp2^{\ARITH}$                        & 387.46                  & 752.12    & 1\,891.61               &   & 13\,458.62 & 14\,164.27 & 16\,461.11 \\
%             $FX\_Log2^{\ARITH}$                        & 181.10                  & 215.64    & 292.53                  &   & 4\,371.94  & 4\,716.72  & 5\,565.20  \\
%             $FX\_Exp^{\ARITH}$                         & 444.02                  & 1\,043.37 & 1\,924.93               &   & 13\,931.69 & 14\,527.76 & 16\,205.35 \\
%             $FX\_Ln^{\ARITH}$                          & 223.29                  & 280.02    & 308.78                  &   & 4\,864.74  & 5\,049.57  & 5\,652.10  \\
%             $FX\_Sqrt^{\ARITH}$                        & 343.14                  & 438.70    & 1\,682.41               &   & 10\,792.22 & 11\,573.25 & 13\,250.91 \\
%             $FX\_Fx2FL^{\ARITH}$                       & 40.91                   & 50.22     & 187.62                  &   & 1\,653.18  & 1\,733.67  & 1\,866.73  \\
%             $FX\_Fx2Int^{\ARITH}$                      & 13.81                   & 17.26     & 26.25                   &   & 336.89     & 338.50     & 341.04     \\
%             \bottomrule
%         \end{tabular}
%     }
% \end{table}
% \FloatBarrier

% \paragraph{Floating-Point Benchmarks}
% \label{para:Floating-PointBenchmarking}


\begin{figure}[ht]
    \centering
    % scatter plot of member decompose-types-bdd-graph-weak member types
    \begin{tikzpicture}[scale=1,transform shape]
        \begin{groupplot}[
                group style={
                        group name=my plots,
                        group size=2 by 5,
                        xlabels at=edge bottom,
                        ylabels at=edge left,
                        horizontal sep=2cm,
                        vertical sep=1.5cm,
                    },
                width=0.5\linewidth
            ]
            %% label = 
            \nextgroupplot[    %place legend outside of diagram. right side, centered vertically.
                % xlabel={Number of Parties},
                % ylabel={Online Runtime (ms)},
                %axis y discontinuity = parallel,
                axis y line = left,
                axis x line = bottom,
                ymin = 0,
                ymax = 2000,
                xmin = 1.5,
                xmax = 5.5,
                grid = both,
                minor tick num = 1,
                width = 7cm,
                height = 3.8cm,
                title=\textbf{(a) addition},
                anchor=north,
                %legend pos = outer north east,
                % legend entries = {Floating-Point BGMW, Floating-Point AGMW},
                % legend cell align = left,
            ]

            \addplot+ table[x=parties, y=FL_circuit] {data/floating_point_add.dat};
            \addplot+ table[x=parties, y=FL_agmw] {data/floating_point_add.dat};

            \nextgroupplot[    %place legend outside of diagram. right side, centered vertically.
                % xlabel={Number of Parties},
                % ylabel={Online Runtime (ms)},
                %axis y discontinuity = parallel,
                axis y line = left,
                axis x line = bottom,
                ymin = 0,
                ymax = 2000,
                xmin = 1.5,
                xmax = 5.5,
                grid = both,
                minor tick num = 1,
                width = 7cm,
                height = 3.8cm,
                title=\textbf{(b) subtraction},
                %legend pos = outer north east,
                % legend entries = {Floating-Point BGMW, Floating-Point AGMW},
                % legend cell align = left,
            ]

            \addplot+ table[x=parties, y=FL_circuit] {data/floating_point_sub.dat};
            \addplot+ table[x=parties, y=FL_agmw] {data/floating_point_sub.dat};


            \nextgroupplot[    %place legend outside of diagram. right side, centered vertically.
                % xlabel={Number of Parties},
                % ylabel={Online Runtime (ms)},
                %axis y discontinuity = parallel,
                axis y line = left,
                axis x line = bottom,
                ymin = 0,
                ymax = 700,
                xmin = 1.5,
                xmax = 5.5,
                grid = both,
                minor tick num = 1,
                width = 7cm,
                height = 3.8cm,
                title=\textbf{(c) mulitplication},
                %legend pos = outer north east,
                % legend entries = {Floating-Point BGMW, Floating-Point AGMW},
                % legend cell align = left,
            ]

            \addplot+ table[x=parties, y=FL_circuit] {data/floating_point_mul.dat};
            \addplot+ table[x=parties, y=FL_agmw] {data/floating_point_mul.dat};
            \nextgroupplot[    %place legend outside of diagram. right side, centered vertically.
                % xlabel={Number of Parties},
                % ylabel={Online Runtime (ms)},
                %axis y discontinuity = parallel,
                axis y line = left,
                axis x line = bottom,
                ymin = 0,
                ymax = 2500,
                xmin = 1.5,
                xmax = 5.5,
                grid = both,
                minor tick num = 1,
                width = 7cm,
                height = 3.8cm,
                title=\textbf{(d) division},
                %legend pos = outer north east,
                % legend entries = {Floating-Point BGMW, Floating-Point AGMW},
                % legend cell align = left,
            ]

            \addplot+ table[x=parties, y=FL_circuit] {data/floating_point_div.dat};
            \addplot+ table[x=parties, y=FL_agmw] {data/floating_point_div.dat};
            \nextgroupplot[    %place legend outside of diagram. right side, centered vertically.
                % xlabel={Number of Parties},
                % ylabel={Online Runtime (ms)},
                %axis y discontinuity = parallel,
                axis y line = left,
                axis x line = bottom,
                ymin = 0,
                ymax =20000,
                xmin = 1.5,
                xmax = 5.5,
                grid = both,
                minor tick num = 1,
                width = 7cm,
                height = 3.8cm,
                title=\textbf{(e) exp2},
                %legend pos = outer north east,
                % legend entries = {Floating-Point BGMW, Floating-Point AGMW},
                % legend cell align = left,
            ]

            \addplot+ table[x=parties, y=FL_circuit] {data/floating_point_exp2.dat};
            \addplot+ table[x=parties, y=FL_agmw] {data/floating_point_exp2.dat};
            \nextgroupplot[    %place legend outside of diagram. right side, centered vertically.
                % xlabel={Number of Parties},
                % ylabel={Online Runtime (ms)},
                %axis y discontinuity = parallel,
                axis y line = left,
                axis x line = bottom,
                ymin = 0,
                ymax = 40000,
                xmin = 1.5,
                xmax = 5.5,
                grid = both,
                minor tick num = 1,
                width = 7cm,
                height = 3.8cm,
                title=\textbf{(f) log2},
                %legend pos = outer north east,
                % legend entries = {Floating-Point BGMW, Floating-Point AGMW},
                % legend cell align = left,
            ]

            \addplot+ table[x=parties, y=FL_circuit] {data/floating_point_log2.dat};
            \addplot+ table[x=parties, y=FL_agmw] {data/floating_point_log2.dat};
            \nextgroupplot[    %place legend outside of diagram. right side, centered vertically.
                % xlabel={Number of Parties},
                % ylabel={Online Runtime (ms)},
                %axis y discontinuity = parallel,
                axis y line = left,
                axis x line = bottom,
                ymin = 0,
                ymax = 14000,
                xmin = 1.5,
                xmax = 5.5,
                grid = both,
                minor tick num = 1,
                width = 7cm,
                height = 3.8cm,
                title=\textbf{(g) square root},
                %legend pos = outer north east,
                % legend entries = {Floating-Point BGMW, Floating-Point AGMW},
                % legend cell align = left,
            ]

            \addplot+ table[x=parties, y=FL_circuit] {data/floating_point_sqrt.dat};
            \addplot+ table[x=parties, y=FL_agmw] {data/floating_point_sqrt.dat};
            \nextgroupplot[    %place legend outside of diagram. right side, centered vertically.
                % xlabel={Number of Parties},
                % ylabel={Online Runtime (ms)},
                %axis y discontinuity = parallel,
                axis y line = left,
                axis x line = bottom,
                ymin = 0,
                ymax = 800,
                xmin = 1.5,
                xmax = 5.5,
                grid = both,
                minor tick num = 1,
                width = 7cm,
                height = 3.8cm,
                title=\textbf{(h) comparison},
                %legend pos = outer north east,
                % legend entries = {Floating-Point BGMW, Floating-Point AGMW},
                % legend cell align = left,
            ]

            \addplot+ table[x=parties, y=FL_circuit] {data/floating_point_lt.dat};
            \addplot+ table[x=parties, y=FL_agmw] {data/floating_point_lt.dat};
            \nextgroupplot[    %place legend outside of diagram. right side, centered vertically.
                % xlabel={Number of Parties},
                % ylabel={Online Runtime (ms)},
                %axis y discontinuity = parallel,
                axis y line = left,
                axis x line = bottom,
                ymin = 0,
                ymax = 15000,
                xmin = 1.5,
                xmax = 5.5,
                grid = both,
                minor tick num = 1,
                width = 7cm,
                height = 3.8cm,
                title=\textbf{(i) fl2int},
                %legend pos = outer north east,
                % legend entries = {Floating-Point BGMW, Floating-Point AGMW},
                % legend cell align = left,
            ]

            \addplot+ table[x=parties, y=FL_circuit] {data/floating_point_fl2int.dat};
            \addplot+ table[x=parties, y=FL_agmw] {data/floating_point_fl2int.dat};

            \nextgroupplot[
                % label style={font=\tiny}, %place legend outside of diagram. right side, centered vertically.
                % xlabel={Number of Parties},
                % ylabel={Online Runtime (ms)},
                %axis y discontinuity = parallel,
                axis y line = left,
                axis x line = bottom,
                ymin = 0,
                ymax = 15000,
                xmin = 1.5,
                xmax = 5.5,
                grid = both,
                minor tick num = 1,
                width = 7cm,
                height = 3.8cm,
                title=\textbf{(j) fl2fx},
                %legend pos = outer north east,
                % legend entries = {Floating-Point BGMW, Floating-Point AGMW},
                % legend cell align = left,
                legend style={at={(0.5,-0.2)},anchor=north},
                % label style={font=\tiny},
                legend to name=UniversalLegend
            ]

            \addplot+ table[x=parties, y=FL_circuit] {data/floating_point_fl2fx.dat};
            \addplot+ table[x=parties, y=FL_agmw] {data/floating_point_fl2fx.dat};

            \legend{\booleanGMW Floating-Point w/o \simd, \arithmeticGMW Floating-Point}
        \end{groupplot}
    \end{tikzpicture}\\
    \ref{UniversalLegend}
    \caption[Total runtime in microseconds for the \booleanGMW and \arithmeticGMW floating-point arithmetic operations.]{Total runtime in microseconds (y-axis) for the \booleanGMW and \arithmeticGMW floating-point arithmetic operations in WAN test encironments with different parties (x-axis). We take the average results over $10$ protocol runs. }
    \label{fig:floatingpointBenchmarks}
\end{figure}
\FloatBarrier



% In \autoref{tab:runtimes_floating_point_circuit}, natural logarithm takes the longest online run-times in both LAN and WAN settings. As the number of parties grows from $3$ to $5$, the run-times of conversion operations to integer and fixed-point increase at most by a factor $3.08\times$ and $3.66\times$ ($SIMD=1$), factor $2.2\times$ and $1.9\times$ ($SIMD=1000$).
% Comparing the online run-times of floating-point arithmetic from \autoref{tab:runtimes_floating_point_circuit} and \autoref{tab:runtimes_floating_point_agmw}, we can see that without applying SIMD, the BGMW-based floating-point arithmetic is faster ($1.3\times-3\times$) than AGMW-based floating-point arithmetic in operations such as addition, subtraction, <, log2 and square root, but slower in all other operations. After applying the SIMD technique, the BGMW-based floating-point arithmetic is $50\times-10911\times$ faster than AGMW-based floating-point arithmetic for all the operations in the LAN setting.
% Therefore, we choose BGMW-based floating-point arithmetic to implement MPC protocols for sampling algorithms.

% \begin{table}
%     \caption{
%         Online run-times in milliseconds (ms) of floating-point operations for the GMW (B) \TODO{and BMR protocol (Y)}. For the entries with $SIMD=1\,000$, we specify the run-time of a single operation amortized over $ 1\,000 $ SIMD values. We take the average over 10 protocol runs in the LAN and WAN environments.
%     }
%     \label{tab:runtimes_floating_point_circuit}
%     % \small
%     \centering
%     \nprounddigits{1} % remove fractional digits in this table
%     \rowcolors{1}{gray!25}{white}
%     \resizebox{\columnwidth}{!}{
%         \begin{tabular}{ l c r r r r r r r r r r}
%             \toprule
%             \hiderowcolors                             &         & \multicolumn{3}{c}{LAN} &           & \multicolumn{3}{c}{WAN}                                              \\
%             \cmidrule{3-5} \cmidrule{7-9}  Parties $N$ & SIMD    &
%             $N{=}2$                                    & $N{=}3$ & $N{=}5$                 &           &
%             $N{=}2$                                    & $N{=}3$ & $N{=}5$                                                                                                    \\ \showrowcolors
%             \midrule
%             $FL\_Add^{\BOOL}$                          & 1       & 158.90                  & 175.47    & 266.47                  &  & 4\,976.43   & 5\,665.71   & 5\,780.94   \\
%             $FL\_Sub^{\BOOL}$                          & 1       & 151.38                  & 167.19    & 252.31                  &  & 5\,053.17   & 5\,599.27   & 5\,970.80   \\
%             $FL\_Mul^{\BOOL}$                          & 1       & 166.46                  & 178.80    & 222.60                  &  & 6\,759.47   & 7\,278.22   & 7\,671.80   \\
%             $FL\_Div^{\BOOL}$                          & 1       & 1\,418.62               & 1\,659.29 & 1\,882.86               &  & 74\,821.42  & 89\,112.86  & 100\,660.43 \\
%             $FL\_Lt^{\BOOL}$                           & 1       & 47.08                   & 52.42     & 132.53                  &  & 1\,638.13   & 1\,751.99   & 1\,840.56   \\
%             $FL\_Exp2^{\BOOL}$                         & 1       & 1\,958.28               & 2\,391.79 & 3\,392.22               &  & 112\,465.43 & 129\,661.24 & 140\,737.75 \\
%             $FL\_Log2^{\BOOL}$                         & 1       & 1\,271.88               & 1\,599.22 & 2\,581.09               &  & 49\,886.75  & 58\,236.84  & 64\,266.20  \\
%             $FL\_Exp^{\BOOL}$                          & 1       & 2\,122.94               & 2\,564.97 & 3\,461.87               &  & 118\,774.84 & 137\,115.63 & 150\,439.77 \\
%             $FL\_Ln^{\BOOL}$                           & 1       & 3\,725.02               & 4\,765.81 & 8\,773.41               &  & 151\,622.29 & 177\,746.42 & 200\,747.65 \\
%             $FL\_Sqrt^{\BOOL}$                         & 1       & 887.72                  & 1\,081.30 & 1\,514.39               &  & 44\,584.20  & 51\,846.96  & 57\,450.22  \\
%             $FL2Int^{\BOOL}$                           & 1       & 327.69                  & 451.41    & 1\,390.65               &  & 12\,943.05  & 14\,093.49  & 16\,199.04  \\
%             $FL2Fx^{\BOOL}$                            & 1       & 453.89                  & 481.33    & 1\,763.47               &  & 19\,657.52  & 21\,243.10  & 24\,070.59  \\
%             \midrule
%             $FL\_Add^{\BOOL}$                          & 1000    & 0.23                    & 0.34      & 0.61                    &  & 4.73        & 5.18        & 5.81        \\
%             $FL\_Sub^{\BOOL}$                          & 1000    & 0.23                    & 0.31      & 0.55                    &  & 4.85        & 5.37        & 5.47        \\
%             $FL\_Mul^{\BOOL}$                          & 1000    & 0.23                    & 0.29      & 0.27                    &  & 6.67        & 7.30        & 7.85        \\
%             $FL\_Div^{\BOOL}$                          & 1000    & 2.45                    & 2.94      & 3.82                    &  & 63.26       & 70.05       & 74.88       \\
%             $FL\_Lt^{\BOOL}$                           & 1000    & 0.05                    & 0.06      & 0.17                    &  & 1.42        & 1.53        & 1.53        \\
%             $FL\_Exp2^{\BOOL}$                         & 1000    & 1.78                    & 2.05      & 2.66                    &  & 84.15       & 96.90       & 105.37      \\
%             $FL\_Log2^{\BOOL}$                         & 1000    & 1.19                    & 1.59      & 2.23                    &  & 42.75       & 48.91       & 52.82       \\
%             $FL\_Exp^{\BOOL}$                          & 1000    & 1.94                    & 2.28      & 2.57                    &  & 90.90       & 104.04      & 113.61      \\
%             $FL\_Ln^{\BOOL}$                           & 1000    & 4.12                    & 5.57      & 7.54                    &  & 128.44      & 146.59      & 163.34      \\
%             $FL\_Sqrt^{\BOOL}$                         & 1000    & 0.14                    & 0.30      & 0.59                    &  & 2.48        & 2.89        & 3.30        \\
%             $FL2Int^{\BOOL}$                           & 1000    & 0.21                    & 0.32      & 0.74                    &  & 10.26       & 11.36       & 12.31       \\
%             $FL2Fx^{\BOOL}$                            & 1000    & 0.33                    & 0.53      & 1.03                    &  & 16.64       & 18.06       & 19.56       \\
%             \bottomrule
%         \end{tabular}
%     }
% \end{table}
% \FloatBarrier

% \begin{table}
%     \caption{
%         Online run-times in milliseconds (ms) of floating-point operations for the GMW (A). We take the average over 10 protocol runs in the LAN and WAN environments. \TODO{implement FL-FL2Fx and FL-FL2Int}
%     }
%     \label{tab:runtimes_floating_point_agmw}
%     % \small
%     \centering
%     \nprounddigits{1} % remove fractional digits in this table
%     \rowcolors{1}{gray!25}{white}
%     \resizebox{\columnwidth}{!}{
%         \begin{tabular}{ l c r r r r r r r r r r}
%             \toprule
%             \hiderowcolors                             & \multicolumn{3}{c}{LAN} &           & \multicolumn{3}{c}{WAN}                                             \\
%             \cmidrule{2-4} \cmidrule{6-8}  Parties $N$ &
%             $N{=}2$                                    & $N{=}3$                 & $N{=}5$   &                         &
%             $N{=}2$                                    & $N{=}3$                 & $N{=}5$                                                                         \\ \showrowcolors
%             \midrule
%             $FL\_Add^{\ARITH}$                         & 216.99                  & 253.25    & 474.08                  &   & 4\,877.91  & 5\,440.09  & 6\,067.92   \\
%             $FL\_Sub^{\ARITH}$                         & 216.63                  & 314.82    & 491.58                  &   & 5\,000.17  & 5\,356.96  & 6\,113.30   \\
%             $FL\_Mul^{\ARITH}$                         & 53.35                   & 75.07     & 84.27                   &   & 1\,248.12  & 1\,263.70  & 1\,553.22   \\
%             $FL\_Div^{\ARITH}$                         & 123.93                  & 168.83    & 301.37                  &   & 3\,173.60  & 3\,344.13  & 3\,485.83   \\
%             $FL\_Lt^{\ARITH}$                          & 66.03                   & 84.35     & 197.75                  &   & 1\,361.24  & 1\,566.96  & 1\,779.53   \\
%             $FL\_Exp2^{\ARITH}$                        & 569.54                  & 953.94    & 1\,784.02               &   & 9\,105.86  & 10\,054.34 & 13\,147.79  \\
%             $FL\_Log2^{\ARITH}$                        & 3\,871.35               & 5\,395.07 & 8\,623.15               &   & 90\,173.12 & 99\,853.84 & 107\,905.28 \\
%             $FL\_Exp^{\ARITH}$                         & 620.11                  & 966.76    & 1\,684.20               &   & 9\,748.59  & 10\,695.65 & 11\,560.34  \\
%             $FL\_Ln^{\ARITH}$                          & 3\,903.48               & 5\,588.28 & 8\,516.19               &   & 90\,034.64 & 99\,444.66 & 108\,395.14 \\
%             $FL\_Sqrt^{\ARITH}$                        & 1\,575.21               & 2\,372.36 & 3\,603.36               &   & 31\,263.47 & 33\,778.76 & 36\,776.32  \\
%             $FL\_FL2Fx^{\ARITH}$                       &                         &           &                         &   &            &            &             \\
%             $FL\_FL2Int^{\ARITH}$                      &                         &           &                         &   &            &            &             \\
%             \bottomrule
%         \end{tabular}
%     }
% \end{table}
% \FloatBarrier

% % \section{Sampling Algorithms}

% \section{Evaluation of MPC Protocols for Differentially Private Mechanisms}
% \label{sec:EvaluationofMPCProtocolsforDifferentiallyPrivateMechanisms}

% \autoref{tab:OverviewofsamplingalgorithmsforgeneratingdiscreteLaplaceGaussianrandomvariable} presents the combinations of different sampling algorithms to generate discrete Laplace random variables and discrete Gaussian random variables as discussed in~\autoref{cha:MPCProtocolsforDifferentiallyPrivateMechanisms}. We implement corresponding protocols in BGMW-based fixed/floating-point arithmetic.
% The benchmark results are listed in \autoref{tab:RuntimesDiscreteLaplaceGaussianProtocols}.
% We found that memory overflow ($>128$ GB) happends for $DLap_1^{\BOOL}$, $DGau_1^{\BOOL}$ and $DGau_2^{\BOOL}$ when we set the failure probability $p<2^{-40}$.
% The reason is that for failure probability $p<2^{-40}$, discrete random variables need a larger number of iterations.
% \TODO{Therefore, we need to reimplement the MPC protocols for $DLap_1$, $DGau_1$ and $DGau_2$ with AGMW-based fixed/floating protocols. }


% \begin{table}
%     \caption{
%         Overview of sampling algorithms for generating discrete Laplace/Gaussian random variables.
%     }
%     \label{tab:OverviewofsamplingalgorithmsforgeneratingdiscreteLaplaceGaussianrandomvariable}
%     % \small
%     \centering
%     \nprounddigits{1} % remove fractional digits in this table
%     \rowcolors{1}{gray!25}{white}
%     \resizebox{\columnwidth}{!}{
%         \begin{tabular}{ l c c c c c c c c c c c}
%             \toprule
%             \hiderowcolors                     & \multicolumn{3}{c}{Sampling Algorithms}       &                                                                                     \\
%             \cmidrule{2-4}     Random Variable & $Algo^{TwoSideGeo}$~\cite{googleDP2019GitHub} & $Algo^{DLap}$~\cite{canonne2020discrete} & $Algo^{DGau}$~\cite{canonne2020discrete}
%             \\ \showrowcolors
%             \midrule
%             $DLap_1$                           & $\bullet$                                     &                                          &                                          \\
%             $DLap_2$                           &                                               & $\bullet$                                &                                          \\
%             $DGau_1$                           & $\bullet$                                     &                                          & $\bullet$                                \\
%             $DGau_2$                           &                                               & $\bullet$                                & $\bullet$                                \\
%             \bottomrule
%         \end{tabular}
%     }
% \end{table}
% \FloatBarrier

% % \paragraph{Discrete Laplace/Gaussian Sampling Algorithms}
% % \label{para:DiscreteLaplaceSamplingAlgorithms}

% \begin{table}
%     \caption{
%         Online run-times in milliseconds (ms) of fixed/floating-point Laplace/Gaussian sampling protocols for the GMW (B). We take the average over 10 protocol runs in the LAN and WAN environments.
%     }
%     \label{tab:RuntimesDiscreteLaplaceGaussianProtocols}
%     % \small
%     \centering
%     \nprounddigits{1} % remove fractional digits in this table
%     \rowcolors{1}{gray!25}{white}
%     \resizebox{\columnwidth}{!}{
%         \begin{tabular}{ l c r r r r r r r r r r}
%             \toprule
%             \hiderowcolors                             & \multicolumn{3}{c}{LAN} &            & \multicolumn{3}{c}{WAN}                                                  \\
%             \cmidrule{2-4} \cmidrule{6-8}  Parties $N$ &
%             $N{=}2$                                    & $N{=}3$                 & $N{=}5$    &                         &
%             $N{=}2$                                    & $N{=}3$                 & $N{=}5$                                                                               \\ \showrowcolors
%             \midrule
%             $DLap_1^{\BOOL,FX}$                        & ---                     & ---        & ---                     &   & ---         & ---         & ---            \\
%             $DLap_1^{\BOOL,FL}$                        & ---                     & ---        & ---                     &   & ---         & ---         & ---            \\
%             $DLap_2^{\BOOL,FX}$                        & 60\,946.33              & 62\,613.41 & 144\,253.44             &   & 893\,460.44 & 947\,818.24 & 1\,040\,754.52 \\
%             $DLap_2^{\BOOL,FL}$                        & 48\,874.17              & 49\,236.48 & 105\,650.76             &   & 730\,002.77 & 782\,549.19 & 861\,691.71    \\
%             $DGau_1^{\BOOL,FX}$                        & ---                     & ---        & ---                     &   & ---         & ---         & ---            \\
%             $DGau_1^{\BOOL,FL}$                        & ---                     & ---        & ---                     &   & ---         & ---         & ---            \\
%             $DGau_2^{\BOOL,FX}$                        & ---                     & ---        & ---                     &   & ---         & ---         & ---            \\
%             $DGau_2^{\BOOL,FL}$                        & ---                     & ---        & ---                     &   & ---         & ---         & ---            \\
%             \bottomrule
%         \end{tabular}
%     }
% \end{table}
% \FloatBarrier




% % \paragraph{Discrete Gaussian Sampling Algorithms}
% % \label{para:DiscreteGaussianSamplingAlgorithms}


% % \begin{table}
% %     \caption{
% %         Online run-times in milliseconds (ms) of fixed/floating-point Laplace sampling protocols for the GMW (B). We take the average over 10 protocol runs in the LAN and WAN environments. \TODO{implement $DGau_1$, $DGau_2$ in AGMW-based fixed/floating-point, as BGMW version memory overflows(>128GB)}
% %     }
% %     \label{tab:RuntimesDiscreteLaplaceProtocols}
% %     % \small
% %     \centering
% %     \nprounddigits{1} % remove fractional digits in this table
% %     \rowcolors{1}{gray!25}{white}
% %     \resizebox{\columnwidth}{!}{
% %         \begin{tabular}{ l c r r r r r r r r r r}
% %             \toprule
% %             \hiderowcolors                             & \multicolumn{3}{c}{LAN} &         & \multicolumn{3}{c}{WAN}                                   \\
% %             \cmidrule{2-4} \cmidrule{6-8}  Parties $N$ &
% %             $N{=}2$                                    & $N{=}3$                 & $N{=}5$ &                         &
% %             $N{=}2$                                    & $N{=}3$                 & $N{=}5$                                                             \\ \showrowcolors
% %             \midrule
% %             $DGau_1^{\BOOL,FX}$                        & ---                     & ---     & ---                     &   & ---       & ---       & --- \\
% %             $DGau_1^{\BOOL,FL}$                        & ---                     & ---     & ---                     &   & ---       & ---       & --- \\
% %             $DGau_2^{\BOOL,FX}$                        & ---                     & ---     & ---                     &   & ---       & ---             \\
% %             $DGau_2^{\BOOL,FL}$                        & ---                     & ---     & ---                     &   & ---       & ---       & --- \\
% %             $DGau_3^{\BOOL,FX}$                        & 614.31                  & 701.92  & ---                     &   & 7\,388.63 & 8\,273.27 & --- \\
% %             $DGau_3^{\BOOL,FL}$                        & 305.42                  & 303.13  & ---                     &   & 6\,324.08 & 7\,329.42 & --- \\
% %             \bottomrule
% %         \end{tabular}
% %     }
% % \end{table}
% % \FloatBarrier







% % \begin{table}
% %     \caption{
% %         Online run-times in milliseconds (ms) for sampling algorithms for the GMW (B). For each entry, we specify the run-time of a single operation amortized over $ 1\,000 $ SIMD values. We take the average over 10 protocol runs in the LAN and WAN environments.
% %     }
% %     \label{tab:runtimes_differentially_private_mechanism}
% %     % \small
% %     \centering
% %     \nprounddigits{1} % remove fractional digits in this table
% %     \rowcolors{1}{gray!25}{white}
% %     \resizebox{\columnwidth}{!}{
% %         \begin{tabular}{ l c r r r r r r r r r r}
% %             \toprule
% %             \hiderowcolors                                                       & \multicolumn{3}{c}{LAN} &            & \multicolumn{3}{c}{WAN}                                               \\
% %             \cmidrule{2-4} \cmidrule{6-8}  Parties $N$                           &
% %             $N{=}2$                                                              & $N{=}3$                 & $N{=}5$    &                         &
% %             $N{=}2$                                                              & $N{=}3$                 & $N{=}5$                                                                            \\ \showrowcolors
% %             \midrule
% %             $FLSymmBino\_Google^{\BOOL}$                                         & 18\,915.60              & 15\,412.45 & 26\,928.51              &   & 394\,237.17 & 435\,113.80 & 470\,800.29 \\
% %             $FxDiscreteGaussian\_noise\_CKS^{\BOOL}$                             & ---                     & ---        & ---                     &   & ---         & ---         & ---         \\
% %             $FxDiscreteGaussian\_noise\_CKS\_with\_DiscreteLaplaceEKMPP^{\BOOL}$ & 614.19                  & 701.75     & ---                     &   & 7\,386.97   & 8\,270.32   & ---         \\
% %             $FLDiscreteGaussian\_noise\_CKS\_with\_DiscreteLaplaceEKMPP^{\BOOL}$ & 304.96                  & 302.50     & ---                     &   & 6\,312.68   & 7\,316.95   & ---         \\
% %             \bottomrule
% %         \end{tabular}
% %     }
% % \end{table}
% % \FloatBarrier

\section{Differentially Private Mechanism Benchmarks}
\label{sec:DifferentiallyPirvateMechanismBenchmarks}
In this section, we present the performance evaluation of the \smpc protocols for differentially private mechanisms.

\paragraph{Laplace Mechanisms.}
Recall that Integer-Scaling Laplace mechanism $M_{ISLap} $ (cf.~\autoref{subsec:ISLap}) is a \textit{secure} implementation of the Laplace mechanism using a re-scaled discrete Laplace random variable to simulate the continuous Laplace random variable.
% In \autoref{cha:MPCProtocolsforDifferentiallyPrivateMechanisms}, we construct two \smpc protocols (\autoref{prot:TwoSideGeometric} and \autoref{prot:DiscreteLap}) for sampling discrete Laplace random variables. However, \autoref{prot:TwoSideGeometric} incurs memory overflow ($>128$GB in \twopc) because its sub-protocol \autoref{prot:GeometricExpBinarySearch} use binary search (with $52$ successive iterations) to generate a geometric random variable. In contrast, the \textbf{FOR} loop in \autoref{prot:DiscreteLap} and its sub-protocol \autoref{prot:GeometricEXP} can be parallelized with \simd that improve the efficiency and decrease the memory overhead. Therefore, we use \autoref{prot:DiscreteLap} to implemented the Integer-Scaling Laplace mechanism $M_{ISLap}$.

In \autoref{cha:MPCProtocolsforDifferentiallyPrivateMechanisms}, we construct two \smpc protocols (\autoref{prot:TwoSideGeometric} and \autoref{prot:DiscreteLap}) for sampling discrete Laplace random variables. However, \autoref{prot:TwoSideGeometric} incurs memory overflow ($>128$GB in \twopc) because its sub-protocol \autoref{prot:GeometricExpBinarySearch} use binary search (with $52$ successive iterations) to generate a geometric random variable. In contrast, the \textbf{FOR} loop in \autoref{prot:DiscreteLap} and its sub-protocol \autoref{prot:GeometricEXP} can be parallelized with \simd that improve the efficiency and decrease the memory overhead.
Nevertheless, we can only generate ten discrete Laplace random variables with \autoref{prot:DiscreteLap} without memory overflow for a single run.
We use \autoref{prot:DiscreteLap} to implemented the Integer-Scaling Laplace mechanism $M_{ISLap}$.


We can see that the Integer-Scaling Laplace mechanism $M_{ISLap}$ with \autoref{prot:DiscreteLap} under \booleanGMW fixed-point is $1.07\times-1.32\times$ slower than that with \booleanGMW floating-point implementation.
The major overhead of \booleanGMW fixed-point is the division operation.

For comparison, we implement the \textit{insecure} Laplace mechanisms $M_{Lap}$~\cite{eigner2014differentially}.
Note that Mironov~\cite{mironov2012significance} showed that $M_{Lap}$~\cite{eigner2014differentially} suffered from floating-point attacks and proposed the snapping mechanism $M_{SM}$ as a solution.

The snapping mechanism $M_{SM}$ has the best total runtime performance and it is $554\times-1373\times $ faster than $M_{ISLap}$ and $1.32\times- 1.62\times$ faster than $M_{Lap}$~\cite{eigner2014differentially}.
However, it is worth to mention that the snapping mechanism $M_{SM}$ introduces additional errors (beyond the necessary amount Laplace noise), that leads to a significant reduction in utility~\cite{Covington2019,googleDP2019}.
In contrast, we could reduce the redundant errors of the Integer-Scaling Laplace mechanism $M_{ISLap}$ by setting an appropriate resolution parameter $r$.

% is $5500\times-9444\times$ slower than the \textit{insecure} floating-point implementation of $M_{Lap}$.


% However, Covington~\cite{Covington2019} proved that $M_{SM}$ introduces additional errors (beyond the necessary Laplace noise), and it leads to a significant reduction in utility.
% In comparison to $M_{SM}$, $M_{ISLap}$ introduces relative low errors by setting appropriate resolution parameter $r$ (cf.~\autoref{sec:integerScalingMechanism}).

\paragraph{Gaussian Mechanisms.}
The Integer-Scaling Gaussian mechanism $M_{ISLap}$ (cf.~\autoref{subsec:IntegerScalingGaussianMechanism}) deploys the sampling protocol \autoref{prot:SymmBinomialLargeN} under \booleanGMW floating-point to simulate the continuous Gaussian random variable.
We can see that the Integer-Scaling Gaussian mechanism $M_{ISGau}$ is $3,27\times-4.11\times$ slower than the Integer-Scaling Laplace mechanism $M_{ISLap}$ under \booleanGMW floating-point.
% Recall that $M_{ISGau}$ realize similar differential privacy protection effect as Gaussian mechanism (cf.~\autoref{def:gaussianMechanism}).
% We implement the Integer-Scaling Gaussian mechanism $M_{ISGau}$ under BGMW-based floating-point that has an online run-times of $18$ seconds in the LAN setting of two computation parties.

\paragraph{Discrete Laplace Mechanisms.}
The discrete Laplace mechanism $M_{DLap}$ deploys the same sampling protocol \autoref{prot:DiscreteLap} as $M_{ISLap}$.
We also implement $M_{DLap}$~\cite{eigner2014differentially} for comparison.
It can be seen, that $M_{DLap}$~\cite{eigner2014differentially} is at least $1764\times$ faster than our implementation of $M_{DLap}$ in \booleanGMW floating-point in the LAN setting. However, the security of $M_{DLap}$~\cite{eigner2014differentially} remains to prove as it applies similar noise generation procedure as $M_{Lap}$~\cite{eigner2014differentially}.

\paragraph{Discrete Gaussian Mechanisms.}
We encounter memory overflow when implement \booleanGMW $M_{DGau}$ with \autoref{prot:DGaussMechanism}. To get a rough estimation fo the runtime, we split \autoref{prot:DGaussMechanism} into several independent parts, save the intermediate result, and run MOTION~\cite{braun2022motion} for multiply times.
As \autoref{tab:runtimes_differentially_private_mechanism} shows, the discrete Gaussian mechanism $M_{DGau}$ takes the longest runtime. To successfully generate a discrete Gaussian random variable, \autoref{prot:DGaussMechanism} needs about $23$ discrete Laplace random variables to be generated in the sub-protocol (cf.~\autoref{prot:DiscreteLap}).


\begin{table}
    \caption{
        Online run-times in milliseconds (ms) for differentially private mechanisms for the GMW (B). We specify the run-time of a single operation amortized over corresponding Parallel Factor (PF) values. We take the average over 10 protocol runs in the LAN and WAN environments. \text{*}: insecure methods.  
    }
    \label{tab:runtimes_differentially_private_mechanism}
    % \small
    \centering
    \nprounddigits{1} % remove fractional digits in this table
    \rowcolors{1}{gray!25}{white}
    \resizebox{\columnwidth}{!}{
        \begin{tabular}{ l r c r r r r r r r r r}
            \toprule
            \hiderowcolors                                       &         &            & \multicolumn{3}{c}{LAN} &            & \multicolumn{3}{c}{WAN}                                              \\
            \cmidrule{4-6} \cmidrule{8-10}  Parties $N$          & PF      & Protocol   &
            $N{=}2$                                              & $N{=}3$ & $N{=}5$    &                         &
            $N{=}2$                                              & $N{=}3$ & $N{=}5$                                                                                                                  \\ \showrowcolors
            \midrule
            % $M_{Lap}$~\cite{eigner2014differentially}                                        & 1000    & $\BOOL,FX$ & 26.09                   & 24.04      & 62.74                   &  & 1\,269.25   & 1\,335.42   & 1\,474.29      \\
            $M_{Lap}$~\cite{eigner2014differentially} \textbf{*} & 1000    & $\BOOL,FL$ & 7.58                    & 8.36       & 11.08                   &  & 197.68      & 223.92      & 248.53      \\
            $M_{SM}$ (\textbf{this work})                        & 1000    & $\BOOL,FL$ & 4.68                    & 5.77       & ---                     &  & 145.70      & 168.74      & 182.60      \\
            $M_{ISLap} $ (\textbf{this work})                    & 10      & $\BOOL,FX$ & 7\,292.91               & 7\,925.98  & ---                     &  & 95\,448.89  & 110\,170.46 & ---         \\
            $M_{ISLap} $ (\textbf{this work})                    & 10      & $\BOOL,FL$ & 5\,770.09               & 5\,981.37  & ---                     &  & 85\,892.69  & 93\,238.29  & ---         \\
            \midrule
            % $M_{Gau}$~\cite{wu2016inherit}                                      & 100     & $\BOOL,FL$ & 29.82                   & 36.71      & 48.36                   &  & 762.88      & 843.09      & 904.58         \\
            $M_{ISGau}$ (\textbf{this work})                     & 1       & $\BOOL,FL$ & 18\,916.06              & 19\,413.08 & 26\,929.39              &  & 394\,248.57 & 435\,126.28 & 470\,813.95 \\
            \midrule
            $M_{DLap}$~\cite{eigner2014differentially}           & 1000    & $\BOOL,FX$ & 2.26                    & 3.26       & ---                     &  & 19.01       & 24.11       & 69.88       \\
            $M_{DLap}$~\cite{eigner2014differentially}           & 1000    & $\BOOL,FL$ & 5.91                    & 6.60       & 9.06                    &  & 144.93      & 163.96      & 177.91      \\
            $M_{DLap}$ (\textbf{this work})                      & 10      & $\BOOL,FX$ & 6\,665.97               & 7\,107.66  & ---                     &  & 93\,155.60  & 101\,739.97 & ---         \\
            $M_{DLap}$ (\textbf{this work})                      & 10      & $\BOOL,FL$ & 5\,394.89               & 5\,749.11  & ---                     &  & 85\,606.52  & 92\,518.90  & ---         \\
            \midrule
            % $M_{Dau}$ \TODO{add other work for comparison}                                   & 1       &            &                         &            &                         &  &             &             &                \\
            % $M_{Dau}$ (\textbf{this work w/~\cite{eigner2014differentially} as subprotocol}) & 50      & $\BOOL,FX$ & 2.03                    & 2.68       & ---                     &  & 20.43       & 27.53       & ---            \\
            % $M_{Dau}$ (\textbf{this work w/~\cite{eigner2014differentially} as subprotocol}) & 50      & $\BOOL,FL$ & 6.51                    & 7.80       & 10.45                   &  & 174.76      & 198.04      & 211.05         \\
            $M_{Dau}$ (\textbf{this work})                       & 1       & $\BOOL,Fx$ & 28\,159.07              & 29\,575.94 & ---                     &  & 562\,198.23 & 617\,524.26 & ---         \\
            $M_{Dau}$ (\textbf{this work})                       & 1       & $\BOOL,FL$ & 24\,906.67              & 26\.350.66 & ---                     &  & 526\.837.85 & 588\,904.17 & ---         \\
            \bottomrule
        \end{tabular}
    }
\end{table}
\FloatBarrier














