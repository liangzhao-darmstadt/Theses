\chapter{General Procedure for MPC-DP Protocols}
\label{cha:ProcedureMPCDP}

% copy to overleaf start from here

% \textbf{To do}:\\
% prelimineary part:
% \begin{itemize}
% 	\item \textit{!suppose MPC, outsource, secret-share, DP is introduced in preliminary part. }
% 	\item For $n \in \mathbb{N}$, $\left[ n\right] $ denotes $\left\{ z\in \mathbb{Z}:1\leq z\leq n\right\} $.
% 	\item For $n \in \mathbb{N}$, $\left[ 0\ldots n\right] $ denotes $\left\{ z\in \mathbb{Z}:0\leq z\leq n-1\right\} $
% 	\item distribution definition (bern, bino, laplace, gaussian, discrete laplace)
% 	\item central limit theory
% 	\item inversion method
% 	\item define infimum
% 	\item add reference and cross reference.
% 	\item rename protocol for consistency
% 	\item define term
% \end{itemize}


\section{MPC-DP Protocols}

\CHANGED{revised based on feedback}

In this chapter, we restate the investigated research problem formally and introduce our new MPC-DP protocol that combines secure multi-party computation protocols and differentially private mechanisms.

Let a dataset $D=\left( D_{1},\ldots, D_{n}\right) $ be distributed among $n$ parties $\left( P_{i}\right) _{i\in \left[ n\right] }$ who do not trust each other, where party $P_{i}$ owns data $D_{i}$. The parties $\left(P_{i}\right)_{i \in \left[ n\right]} $ wish to jointly compute a public function $f$ on their private inputs $\left(pre\left(D_i\right)\right)_{i \in \left[ n\right]} $ and get the result of $f\left(pre\left(D_1\right),\ldots,pre\left(D_n\right)\right)$, where $pre$ is a pre-processing function executing on their local data $\left(D_i\right)_{i \in \left[ n\right]} $. The parties want to achieve both computional privacy (i.e., each party's input is kept secret), output privacy (i.e., an adversary cannot infer sensitive information of parties), and obatin a result with reasonable accuracy.
% \begin{enumerate}
% 	\item The computation of $f$ should reveal nothing but the result.
% 	\item The computation result of $f\left(pre\left(D_1\right),\ldots,pre\left(D_N\right)\right)$ should be identical to the scenario, where a trusted server has access to the whole dataset $D$ and computes $f\left( pre\left(D\right) \right)$ locally, i.e., $f\left(pre\left(D_1\right),\ldots,pre\left(D_N\right)\right)=f\left( pre(D)\right)$.
% 	\item The result has to satisfy the differential privacy by adding it with noise, i.e., an adversary cannot infer too much information from the noisy result.
% 	\item The noisy result can still provide statistically useful information.
% \end{enumerate}

To achieve the computional privacy, the parties can deploy a MPC protocol $\Uppi^{f}$ which takes $\left(pre\left(D_i\right)\right)_{i \in \left[ n\right]} $ as input and reveals only the computation result. For the last two requirements, we design a MPC-DP protocol which combines $\Uppi^{f}$ with a differentially private mechanism. Note that in following, we omit the data pre-processing procedure $pre\left(\cdot\right) $ for simplicity and use $f\left( D\right)$ to represent $f\left(D_1,\ldots,D_n\right)$.

Before describing our MPC-DP protocol, let's consider a trival example that combines MPC protocols with a DP mechanism. Suppose the parties $\left(P_{i}\right)_{i \in \left[ n\right]} $ wish to calculate the sum of their local data $\left(D_{i}\right)_{i \in \left[ n\right]} $ with a public known function $f\left(D_1,\ldots,D_n\right)=\sum_{j=1}^{n} D_j$, and $f$ has $\ell_2$-\textit{sensitivity} $\Delta^{\left(f\right) } _2=1$ (cf.~\autoref{def:sensitivity}). Meanwhile, the parties require that only the computation result of $f$ should be revealed and differential privacy must be satisfied. To achieve this, each party first adds noise $r_i$ to its local data $D_i$ and determines $ y_{i}=D_{i} +r_{i}$. Recall that in \autoref{def:gaussianMechanism}, we have proved that $ y_{i}=$ satisfies $\left(\varepsilon ,\delta \right) $-DP for $r_i \sim \mathcal{N}\left( 0,\sigma ^{2}\right) $, where $\sigma ^{2}=\frac{2}{\varepsilon ^2}\cdot\ln\left(\frac{1.25}{\delta }\right)$ and $i\in \left[n\right] $. Then, the parties jointly run the MPC protocol $\Uppi^{f}$ with their private inputs $y_i$ to compute function $f$. After reconstruction they get:

\begin{equation}
	\begin{split}
		f\left(y_1,\ldots,y_n\right)&=\sum_{j=1}^{n} y_j\\
		&=\sum_{j=1}^{n}\left( D_j+ r_j\right).
	\end{split}
\end{equation}

Because of the infinite divisibility of the normal distribution~\cite{patel1996handbook}, we have $\sum_{j=1}^{n} r_j \sim \mathcal{N}\left( 0,\sigma_{sum} ^{2}\right) $ for $\sigma_{sum} ^{2}=\sum ^{n}_{j=1}\sigma ^{2}$. Therefore, the computation result $f\left(y_1,\ldots,y_n\right)$ satisfies $\left(\frac{\varepsilon}{\sqrt{n}},\delta \right) $-DP.

In general, the parties in above example can achieve $\left(\frac{\varepsilon}{\sqrt{n}},\delta \right) $-DP assuming no corrupted parties. However, in realistic scenarios, corrupted parties may try to extract information of certain parties by weaking the achieved differential privacy protection level.
After executing above steps honestly and obtain the computation result $f\left(y_1,\ldots,y_n\right)$, the corrupted parties subtract the noise they have added from $f\left(y_1,\ldots,y_n\right)$. In the worst case, i.e., $n-1$ parties are corrupted, the differential privacy guarantee of $f\left(y_1,\ldots,y_n\right)$ reduces from $\left(\frac{\varepsilon}{\sqrt{n}},\delta \right) $-DP to $\left(\varepsilon,\delta \right) $-DP. If, however, $\left(\frac{\varepsilon}{\sqrt{n}},\delta \right) $-DP is required, one solution is that each party adds \textit{more} noise,.e.g, $r_i^{\prime} \sim \mathcal{N}\left( 0,n\cdot \sigma ^{2}\right) $, locally before running $\Uppi^{f}$. However, this can lead to a severely reduced accuracy and, thus, utility.

To summarize, a well-designed MPC-DP protocol requires a more careful integration of MPC protocols and DP mechanisms. To guarantee the both the accuracy and privacy of the MPC-DP protocol, we define several requirements,that is based on the comparision to the centralized data server scenario, i.e., a trusted server that has access to all the parties' data $D$ locally computes $f\left( D\right) $ and adds noise for DP:

\begin{enumerate}
	\item The MPC-DP protocol should achieve the same privacy protection level as in the centralized data server scenario, and the amount of noise $r$ (in above example: $r=\sum_{j=1}^{n} r_j$) should be no more than that in the centralized data server scenario.
	\item The differential privacy guarantee of the MPC-DP protocol should not degenerate in the presence of corrupted parties.
\end{enumerate}

We design our MPC-DP protocol that can be used directly among $n$ parties, or in an outsourcing scenario, i.e., an arbitrary amount of parties secret share their private inputs to $N$ ($N\ll n$) non-colluding computing parties. Then, the computing parties execute the $N$-party MPC-DP protocol, where they compute $f$ and add noise shares to get the noisy secret-shared results. Finally, the noisy secret-shared results are sent back to the parties for reconstruction. We assume semi-honest adversaries that follow the protocol specifications but wish to infer additional information.

% Specifically, for $i\in \left[ n\right]$, party $P_{i} $ secret shares his inputs $D_i$ to $N$ computing parties $P_{0},\ldots,P_{N}$. For $j\in \left[ N\right]$, computing party $P_j$ runs MPC protocols for function $f$ with other computing parties using his secret-shared value $\left\langle D_1 \right\rangle _{j},\ldots,\left\langle D_n \right\rangle _{j} $ (received from $n$ parties) and get $\left\langle f\left(D\right) \right\rangle _j$. Then, computing party $P_j$ runs the MPC protocols for distributed noise generation (DNG) with other computing parties and get a secret-shared noise $\left\langle r\right\rangle_j $. Finally, computing party $P_j$ perturbs $\left\langle f\left(D\right) \right\rangle _j$ by computing $\left\langle M\left(D\right)\right\rangle_j=\Uppi^{Perturbation}\left(\left\langle f\left(D\right) \right\rangle_j+\left\langle r\right\rangle_j\right) $ and sends it to the parties responsible for the reconstruction. Note that during the distributed noise generation, each computing party only gets a secret-shared noise $\left\langle r\right\rangle $ and even $N-1$ colluding computing parties cannot reconstruct the noise $r$. After the reconstruction, only the perturbed (noisy) result $M\left(D\right)=\text{Perturb}\left(f\left(D\right)+r\right)  $ is revealed which computes function $f$ and guarantees differential privacy. More important is that we prevent excessive noise and ensure utility since the noise $r$ is generated by all the parties instead of by each party independently (as in the above example).



~\autoref{prot:MPC-DP} describes the general procedure of our MPC-DP protocol. Note that the parties first calculate $\left\langle f\left(D\right)\right\rangle =\Uppi^{f }\left(\left\langle D_1 \right\rangle ,\ldots,\left\langle D_n \right\rangle\right) $ and add the distributed generated noise $\left\langle r\right\rangle$ (output perturbation) to guarantee differential privacy. In this work, we assume that each party has computed $\left\langle f\left(D\right) \right\rangle$ already and focus more on the process of $\Uppi^{DNG }$, i.e., distributed noise generation (DNG).

\begin{protocol}[tbh!]
	\centering
	\fbox{
		\pseudocode[space=none, syntaxhighlight=auto, addkeywords={Protocol, Input, Output},linenumbering, skipfirstln, head=\textbf{Protocol: $\Uppi^{MPC-DP}\left(\left\langle D_1 \right\rangle ,\ldots,\left\langle D_n \right\rangle\right) $}]{
			\textbf{Input: $\left\langle D_1 \right\rangle ,\ldots,\left\langle D_n \right\rangle$ } \pcskipln \\
			\textbf{Output: $\left\langle M\left(D\right)\right\rangle=\left\langle f\left(D\right) \right\rangle +\left\langle r\right\rangle$} \\
			\text{Parties run $\left\langle f\left(D\right)\right\rangle =\Uppi^{f }\left(\left\langle D_1 \right\rangle ,\ldots,\left\langle D_n \right\rangle\right) $.}\\
			\text{Parties run $\left\langle r\right\rangle =\Uppi^{DNG } $.}\\
			\text{Parties run $\left\langle M\left(D\right)\right\rangle=\Uppi^{Perturb}\left(\left\langle f\left(D\right) \right\rangle,\left\langle r\right\rangle\right) $ and output $\left\langle M\left(D\right)\right\rangle$.}
		}}
	\caption{General framework for MPC-DP protocol.}
	\label{prot:MPC-DP}
\end{protocol}
\FloatBarrier



\section{Related Works}

\TODO{Related Works, unsecure MPC-DP protocols}
% % -----------------------------------------------------------------------
% % algorithms for generating noise
% \section{Algorithms for Noise Generation}
% In this section, we first introduce essential algorithms for generating probability distributions. Then we describe the algorithms for generating Gaussian, Laplace, Finite-range discrete Laplace, and discrete Laplace noise.

% \subsection{Basic Algorithms}

% \subsubsection{Algorithm for Binomial Distribution}
% Algorithm $Algo^{Binomial}$ converts a Bernoulli random variable to a binomial random variable. The input of $Algo^{Binomial}$ is bit string $s$ that takes on values 0 or 1 with equal probability $p=\frac{1}{2}$. Based on the fact that the sum of mutually independent Bernoulli random variables $X_{1},\ldots, X_{d}\sim Bernoulli\left( \frac{1}{2}\right) $ is a binomial random variable $Y=\sum ^{d}_{i=1}X_{i}\sim Binomial\left( d,\frac{1}{2}\right) $, we get a binomial random variable sampled from $ Binomial\left(d, \frac{1}{2}\right)$.
% \begin{protocol}[tbh!]
% 	\centering
% 	\fbox{
% 	\pseudocode[space=none, syntaxhighlight=auto, addkeywords={Algorithm, Input, Output},linenumbering, skipfirstln, head=\textbf{Algorithm: $Algo^{Binomial}$}]{
% 	\textbf{Input: $s=\left( s_{1},\ldots ,s_{d}\right) \in \left\{ 0,1\right\} ^{d} $} \pcskipln \\
% 	\textbf{Output: $bino \sim Binomial\left(d, \frac{1}{2}\right)$} \\
% 	\text{$bino=\sum ^{d}_{i=1}s_{i} $.}
% 	}}
% 	\caption{Algorithm for binomial random variables.}
% 	\label{algo:berntobinomial}
% \end{protocol}
% \FloatBarrier

% \subsubsection{Algorithm for Uniform Distribution}
% \begin{theorem}[Inversion-of-f Method for Monotone Densitiy~{\cite[Theorem 2.1]{Devroye1986}}]
% 	\label{theorem:inversionfmethod}
% 	Let $F$ be a continuous distribution function on $\mathbb{R}$ with inverse $F^{-1}$ defined by
% 	\[F^{-1}\left( u\right) =\inf \{ x:F\left( x\right) =u,0 <u < 1\} \]
% 	If $U$ is a $Uniform\left( 0,1\right)$ random variate, then $F^{-1}\left( U\right)$ has a distribution function $F$. Also, if $X$ has distribution function $F$ , then $F\left( X\right) $ is uniformly distributed on $F\left( x\right) $.
% \end{theorem}

% Algorithm $Algo^{Uniform}$ uses the result of theorem \autoref{theorem:inversionfmethod} to generate a uniform random variable $uni \sim Uniform\left(0, 1\right)$.
% \begin{protocol}[tbh!]
% 	\centering
% 	\fbox{
% 		\pseudocode[space=none, syntaxhighlight=auto, addkeywords={Algorithm, Input, Output},linenumbering, skipfirstln, head=\textbf{Algorithm: $Algo^{Uniform}$}]{
% 			\textbf{Input: $s \sim F$} \pcskipln \\
% 			\textbf{Output: $uni \sim Uniform\left(0, 1\right)$} \\
% 			\text{$uni=F\left( s\right)$.}
% 		}}
% 	\caption{Algorithms for uniform random variable.}
% 	\label{algo:invtouniform}
% \end{protocol}
% \FloatBarrier

% \subsubsection{Algorithms for Biased Bit}
% Algorithm $Algo^{BiasedBit\left(\alpha\right)}$ generates a biased bit $b$ with statistical distance at most $2^{-d}$ to a Bernoulli random variable $bern\sim Bernoulli\left( \alpha\right) $. The algorithm is based on the folklore method (???ref): Given $d$ random bits $s_{1},\ldots ,s_{d}$ and find the minimum index $j$ such that $s_{j}\neq \alpha _{j}$ and output $b=1-s_j$, where $\alpha _{j}$ is $j$-th most significant bit in the binary expansion of $\alpha$.

% \begin{protocol}[tbh!]
% 	\centering
% 	\fbox{
% 	\pseudocode[space=none, syntaxhighlight=auto, addkeywords={Algorithm, Input, Output},linenumbering, skipfirstln, head=\textbf{Algorithm: $Algo^{BiasedBit\left(\alpha\right)}$}]{
% 	\textbf{Input: $s=\left( s_{1},\ldots ,s_{d}\right) \in \left\{ 0,1\right\} ^{d} $, $\alpha=\left(\alpha_1, \ldots , \alpha_d\right)$.} \pcskipln \\
% 	\textbf{Output: $b$} \\
% 	\text{$j=\min \{ k:s_{k}\neq \alpha _{k},k\in \left[ d\right] \} $.} \\
% 	\text{$b=1-s_j$.}
% 	}}
% 	\caption{Algorithm for biased bit.}
% 	\label{algo:biasedbit}
% \end{protocol}
% \FloatBarrier


% \subsection{Algorithms for Noise Generation}
% \subsubsection{Algorithm for Gaussian Noise}
% In algorithm $Algo^{Gaussian}$, we generate a gaussian random variable $gau$ using the convergence property of binomial distribution~\cite{rastogi2010differentially}.
% \begin{theorem}[Convergence of Binomial to Normal~{\cite[Example 10.3.2]{athreya2006measure}}]
% 	Let $X_{n}\sim Binomial\left( N_{n},p_{n}\right) $ for all $n\geq 1$. Suppose that as $n\to \infty $, $N_{n}\to \infty $ and $s_{n}^{2}\equiv N_{n}p_{n}\left( 1-p_{n}\right) \rightarrow \infty $. Then
% 	\[Z_{n}\equiv \frac{X_{n}-N_{n}p_{n}}{s_{n}} \sim Gaussian\left( 0,1\right) \]
% 	where $\to ^{d}$ denotes converge to a probability distribution.
% \end{theorem}

% \begin{protocol}[tbh!]
% 	\centering
% 	\fbox{
% 	\pseudocode[space=none, syntaxhighlight=auto, addkeywords={Algorithm, Input, Output},linenumbering, skipfirstln, head=\textbf{Algorithm: $Algo^{Gaussian}$}]{
% 	\textbf{Input: $s=\left( s_{1},\ldots ,s_{d}\right) \in \left\{ 0,1\right\} ^{d} $} \pcskipln \\
% 	\textbf{Output: $gau \sim Gaussian\left(0, \sigma^{2}\right)$} \\
% 	\text{$bino=Algo^{Binomial}\left(s\right)$.}\\
% 	\text{$gau=\frac{\sigma\left( bino-\frac{d}{2}\right) }{\frac{\sqrt{d}}{2}}$.}
% 	}}
% 	\caption{Algorithm for gaussian noise based on random bits.}
% 	\label{algo:binomialtogauss}
% \end{protocol}
% \FloatBarrier

% \subsubsection{Algorithm for Laplace Noise}
% \begin{prop}
% 	\label{prop:gausstolap}
% 	Let $Y_{i}\sim Gaussian\left( 0,\sigma ^{2}\right) $ for $i\in \left\{ 1,2,3,4\right\} $ be four Gaussian random variables. Then $Z=Y_{1}^{2}+Y_{2}^{2}-Y_{3}^{2}-Y_{4}^{2}$ is a $Laplace\left( 2\sigma^{2}\right) $ random variable~\cite{rastogi2010differentially}.
% \end{prop}
% In algorithm $Algo^{Laplace}_{Binomial}$, we first convert four random bit strings to four binomial random variables, then use proposition \autoref{prop:gausstolap} to generate a Laplace random variable.

% \begin{protocol}[tbh!]
% 	\centering
% 	\fbox{
% 	\pseudocode[space=none, syntaxhighlight=auto, addkeywords={Algorithm, Input, Output},linenumbering, skipfirstln, head=\textbf{Algorithm: $Algo^{Laplace}_{Binomial}$}]{
% 	\textbf{Input: $s=\left( s_{t1},\ldots ,s_{td}\right) _{t\in \left\{ 1,2,3,4\right\}} \in \left\{ 0,1\right\} ^{4\times d} $} \pcskipln \\
% 	\textbf{Output: $lap \sim Laplace\left(2 \sigma^2\right)$} \\
% 	\text{$gau_t = Algo^{Gaussian}\left(s_{t1},\ldots ,s_{td}\right)$ for $t\in \left\{ 1,2,3,4\right\} $.} \\
% 	\text{$lap=gau_1^2+gau_2^2-gau_3^2-gau_4^2 $.}
% 	}}
% 	\caption{Algorithm for Laplace noise based on random bits.}
% 	\label{algo:binomialtolap}
% \end{protocol}
% \FloatBarrier

% % mathematical result needs to be verified
% Another way to generate a Laplace random variable is using algorithm $Algo^{Laplace}_{Uniform}$, which converts two uniform random variables $ u_1$, $u_2$ to a Laplace random variable according to the mathematical results (ref Handbook of Mathematical Functions with For- mulas, Graphs, and Mathematical Tables, Non-Uniform Random Variate Generation). Since $Laplace\left( b\right) =Exp\left( \frac{1}{b}\right) -Exp\left( \frac{1}{b}\right) $, where the exponential distribution $Exp$ can be generated by $Exp\left( b^{\prime}\right) =\frac{-\ln u}{b^{\prime}}$ and $u \sim Uniform\left( 0,1\right) $ .

% \begin{protocol}[tbh!]
% 	\centering
% 	\fbox{
% 		\pseudocode[space=none, syntaxhighlight=auto, addkeywords={Algorithm, Input, Output},linenumbering, skipfirstln, head=\textbf{Algorithm: $Algo^{Laplace}_{Uniform}$}]{
% 			\textbf{Input: $u_1$, $u_2 \sim Uniform\left( 0,1\right) $} \pcskipln \\
% 			\textbf{Output: $lap \sim Laplace\left(b\right)$} \\
% 			\text{$lap=b\left( \ln u_{1}+\ln u_{2}\right) $.}
% 		}}
% 	\caption{Algorithm for Laplace noise based on uniform distribution.}
% 	\label{algo:uniformtolap}
% \end{protocol}
% \FloatBarrier

% \subsubsection{Algorithm for Finite-range Discrete Laplace Noise}
% \begin{definition} Finite-range discrete Laplace distribution (FDL)~\cite{eriguchi2021efficient} is defined as
% 	\begin{equation}
% 		\Pr[ L= k] =
% 		\begin{cases}
% 			p^{\left| k\right| }\left(\frac{1-p}{1+p}\right), & if \left| k\right| <N, \\
% 			\frac{p^{N}}{1+p}, & if\left| k\right| =N, \\
% 			0, & otherwise.
% 		\end{cases}
% 	\end{equation}
% \end{definition}

% Algorithm $Algo^{FDL\left(p,N \right)}$ first generates $N-1$ biased bits with algorithm $Algo^{BiasedBit\left(\alpha\right)}$, then generate a FDL random variable based on the biased bits. For $FDL\left(p,N \right)$ with $0 <p <1$, set $\alpha_{0}=\frac{1-p}{1+p}$, $\alpha_i=1-p$ for $\left[ N-1\right] $. Then using $\sigma \in Uniform\left( -1,1\right) $ and $b_{i}\sim Bernoulli\left( \alpha _i\right) $, we have $\sigma l\sim FDL\left( p,N\right) $, where $l$ is the smallest index such that $b_{l}=1$. If $l$ not exist, we set $l=N$.

% \begin{protocol}[tbh!]
% 	\centering
% 	\fbox{
% 		\pseudocode[space=none, syntaxhighlight=auto, addkeywords={Algorithm, Input, Output},linenumbering, skipfirstln, head=\textbf{Algorithm: $Algo^{FDL\left(p,N \right)}${~\cite{eriguchi2021efficient}}}]{
% 			\textbf{Input: $s=\left({\left( s_{1}^{\left(j\right)},\ldots ,s_{d}^{\left(j\right)}\right) }_{j\in \left[ N-1\right] },\sigma \right)$, $\alpha =\left( \alpha_0,\ldots ,\alpha_{N-1}\right) $} \pcskipln \\
% 			\textbf{Output: $k \sim FDL\left(p,N\right)$} \\
% 			\text{Run algorithm $Algo^{BiasedBit\left(\alpha_{j}\right)}$ with inputs $s_j=\left( s_{1}^{\left(j\right)},\ldots s_{d}^{\left(j\right)}\right) $ and obtain $ b_{j} $ for $j\in \left[ N-1\right] $.}\\
% 			\text{Calculate $ c_{j} = \vee_{k=1}^j b_{k} $ for $j\in \left[ N-1\right] $.}\\
% 			\text{Calculate $ l =N-\sum ^{N-1}_{j=0} c_j$.} \\
% 			\text{Calcualte $ k = \sigma l $.}
% 		}}
% 	\caption{Algorithms for Finite-range discrete laplace noise based on biased bits.}
% 	\label{algo:biasedbitstoFDL}
% \end{protocol}
% \FloatBarrier

% \subsubsection{Algorithm for Discrete Laplace Noise}
% % mathematical result needs to be verified
% According to the mathematical results (ref Non-Uniform Random Variate Generation, A Discrete Analogue of the Laplace Distribution) algorithm $Algo^{DLaplace}_{Uniform}$ converts two uniform random variables $ u_1$, $u_2$ to a discrete laplace random variable. First, a uniform random variable $u$ is converted to an exponential random variable with $Exp\left( b^{\prime}\right) =\frac{-\ln u}{b^{\prime}}$, then the exponential random variable is transformed to a geometric random variable $Geo\left( b^{\prime}\right) =\big \lfloor Exp\left( -\ln \left( 1-b^{\prime}\right) \right) \big \rfloor $. Finally, we get the discrete Laplace random variable $DLaplace\left( b\right) =Geo\left( 1-b\right) -Geo\left( 1-b\right) $.

% \begin{protocol}[tbh!]
% 	\centering
% 	\fbox{
% 		\pseudocode[space=none, syntaxhighlight=auto, addkeywords={Algorithm, Input, Output},linenumbering, skipfirstln, head=\textbf{Algorithm: $Algo^{DLaplace}_{Uniform}$}]{
% 			\textbf{Input: $u_1$, $u_2 \sim Uniform\left( 0,1\right) $} \pcskipln \\
% 			\textbf{Output: $lap \sim DLaplace\left(b\right)$} \\
% 			\text{$ dlap =\bigg\lfloor \frac{1}{\ln b} \ln u_{1} \bigg\rfloor -\bigg\lfloor \frac{1}{\ln b} \ln u_{2} \bigg\rfloor \sim DLaplace\left(b\right) $.}
% 		}}
% 	\caption{Algorithms for discrete laplace noise based on uniform distribution.}
% 	\label{algo:uniformtodLap}
% \end{protocol}
% \FloatBarrier

% % --------------------------------------------------------------------------
% % framework for MPC-DP protocols
% \section{Framework for MPC-DP Protocols}
% \label{sec:frameworkMPCDP}
% Protocol $\Uppi^{MPC-DP}$ consists of a function $f$ computing with parties' private data $x$ and a function $h$ generating noise with random inputs $s$. In the outsourcing scenario, $x$ represents the secret-shared values sent by partys. In this section~\autoref{sec:frameworkMPCDP}, we mainly describe how to realize the algorithms of noise generation $h$ in the MPC. In the following protocol descriptions, the party index $i$ is omitted when not affecting the understanding.
% % (To do: security analysis regarding adversary's view)

% \begin{protocol}[tbh!]
% 	\centering
% 	\fbox{
% 		\pseudocode[space=none, syntaxhighlight=auto, addkeywords={Protocol, Input, Output},linenumbering, skipfirstln, head=\textbf{Protocol: $\Uppi^{MPC-DP}$}]{
% 			\textbf{Input: $x \in \mathbb{R}^n$, $s \in S^n$} \pcskipln \\
% 			\textbf{Output: $M\left( x\right) =f\left( x\right) +h\left( s\right) $} \\
% 			\text{The parties run $\Uppi^f$ with inputs $x$ and obtain $f\left( x\right)$.} \\
% 			\text{The parties run $\Uppi^h$ with inputs $s$ and obtain $h\left( s\right)$.} \\
% 			\text{The parties compute $\langle M\left(x\right) \rangle =\langle f\left(x\right) \rangle+\langle h\left(s\right) \rangle$.} \\
% 			\text{The parties reconstruct $M\left( x\right)$ and output it.}
% 		}}
% 	\caption{Framework for MPC-DP protocols.}
% 	\label{prot:MPCDP}
% \end{protocol}
% \FloatBarrier

% % protocol for generating noise distributely
% % TODO: define number field for integer, floating point, fixed point
% % ---------------------------------------------------------------------

% \subsection{Building Blocks for Noise Generation}
% In this section, we first introduce the MPC protocols to generate shares of random bit strings and shares of biased bits among parties.
% % random bits
% % \subsubsection{Protocol for Random Bits}
% % Protocol $\Uppi^{RandBits}_d$ is a basic building block for distributed noise generation. Each party first locally generate $d$ random bits $s=\left( s_{1},\ldots ,s_{d}\right)$ , where each bit takes on values 0 or 1 with equal probability $p=\frac{1}{2}$. Then the parties share each bit of the bit string with other parties. Finally, each party locally $XOR$ (?needs to be defined) all the secret-shared bits ${\langle u_{\ell}\rangle} =\oplus _{ k\in \left[ n\right] } \langle s_{k\ell} \rangle$, where $l\in \left[ d\right] $. Based on the fact that the sum of mutually independent Bernoulli random variables $X_{1},\ldots X_{d}\sim Bernoulli\left( \frac{1}{2}\right) $ is a binomial random variable $Y=\sum ^{d}_{i=1}X_{i}\sim Binomial\left( d,\frac{1}{2}\right) $, $U=\left( u_{i1},\ldots ,u_{id}\right) _{i\in \left[ n\right] }$ correspond to the shares of a random variable drawn from distribution $Binomial\left(d,\frac{1}{2}\right)$.
% % % (To do:security analysis regarding adversary's view)
% % % random bits
% % \begin{protocol}[tbh!]
% % 	\centering
% % 	\fbox{
% % 	\pseudocode[space=none, syntaxhighlight=auto, addkeywords={Protocol, Input, Output},linenumbering, skipfirstln, head=\textbf{Protocol: $\Uppi^{RandBits}_d$}]{
% % 	\textbf{Input: $s=\left( s_{i1},\ldots ,s_{id}\right) _{i\in \left[ n\right] }\in \left( \left\{ 0,1\right\} ^{d}\right) ^{n}$} \pcskipln \\
% % 	\textbf{Output: $U=\left( u_{i1},\ldots ,u_{id}\right) _{i\in \left[ n\right] }\in \left( \left\{ 0,1\right\} ^{d}\right) ^{n}$} \\
% % 	\text{Each party shares $s_{l}$ with other parties for $l \in \left[ d\right]$.} \\
% % 	\text{The parties compute ${\langle u_{l}\rangle} = \oplus _{ k\in \left[ n\right] }\langle s_{kl} \rangle $ and output ${\langle u_{l}\rangle}$ for $l \in \left[ d\right]$.}
% % 	}}
% % 	\caption{Protocol for random bits.}
% % 	\label{prot:randbits} % change label
% % \end{protocol}
% % \FloatBarrier



% Protocol $\Uppi^{RandInt}_d$ can achieve the same purpose as protocol $\Uppi^{RandBits}_d$ and without communications between parties. First, each party locally generate a random $d$-bit integer $s$ and convert $s$ into a $d$-length random bit string $U=\left( u_{1},\ldots ,u_{d}\right)$. This protocol can only be applied for MPC protocols with arithmetic sharing.

% % random integer
% \begin{protocol}[tbh!]
% 	\centering
% 	\fbox{
% 	\pseudocode[space=none, syntaxhighlight=auto, addkeywords={Protocol, Input, Output},linenumbering, skipfirstln, head=\textbf{Protocol: $\Uppi^{RandInt}_d$}]{
% 	\textbf{Input: $s=\left( s_{i}\right) _{i\in \left[ n\right] }\in \mathbb{Z}_{2^d}^{n}$} \pcskipln \\
% 	\textbf{Output: $U=\left( u_{i1},\ldots ,u_{id}\right) _{i\in \left[ n\right] }\in \left( \left\{ 0,1\right\} ^{d}\right) ^{n}$} \\
% 	\text{Each party locally convert integer $s$ to a $d$-length bit string ${\langle u_{il}\rangle}$ for $l \in \left[ d\right]$, $i \in \left[ n\right]$.} \\
% 	\text{Each party output ${\langle u_{l}\rangle}$ for $l \in \left[ d\right]$.}
% 	}}
% 	\caption{Protocol for random integer without interactions.}
% 	\label{prot:randint}
% \end{protocol}
% \FloatBarrier

% % biased bit
% \subsubsection{Protocol for Biased Bit}
% Protocol $\Uppi^{BiasedBit\left(\alpha \right)}$ generating a biased bit with statistical distance less than $2^{-d}$ to $Bernoulli\left(\alpha\right)$ using the algorthm $Algo^{BiasedBit\left(\alpha\right)}$ in the MPC. Based on the type of secret sharing, The parties can also use $\Uppi^{FastRandBits}_d$ or $\Uppi^{RandInt}_d$ to generate random bit string share.

% \begin{protocol}[tbh!]
% 	\centering
% 	\fbox{
% 	\pseudocode[space=none, syntaxhighlight=auto, addkeywords={Protocol, Input, Output},linenumbering, skipfirstln, head=\textbf{Protocol: $\Uppi^{BiasedBit\left(\alpha \right)}$}]{
% 	\textbf{Input: $s=\left( s_{i1},\ldots ,s_{id}\right) _{i\in \left[ n\right] }\in \left( \left\{ 0,1\right\} ^{d}\right) ^{n}$, $\alpha=\left( \alpha_1,\ldots ,\alpha_d\right)$} \pcskipln \\
% 	\textbf{Output: $\left(\langle b_{i} \rangle \right)_{i \in \left[n\right]}$} \\
% 	\text{The parties run $\Uppi^{RandBits}_d$ with inputs $s=\left( s_{i1},\ldots s_{id}\right) _{i\in \left[ n\right] }$ and obtain $\langle u_{l}\rangle $ for $l \in \left[ d\right]$.}\\
% 	\text{Each party locally computes $\langle c_{l}\rangle =\langle u_{l}\oplus \alpha _{l}\rangle$ for $l \in \left[ d\right]$. }\\
% 	\text{The parties compute $\langle e_{l}\rangle =\langle \vee_{k=1}^{l}c_{k}\rangle$ for $l \in \left[ d\right]$.} \\
% 	\text{Each party locally computes $\langle f_{l}\rangle =\langle e_{l}\rangle -\langle e_{l-1}\rangle$ for $l \in \left[ d\right]$ and $e_{0}=0$.} \\
% 	\text{The parties compute $\langle f_{l}u_{l}\rangle $ for $l \in \left[ d\right]$.} \\
% 	\text{The parties compute $\langle b\rangle =1-\sum ^{d}_{i=1} \langle f_{l}u_{l}\rangle$ and output $\langle b\rangle $.}
% 	}}
% 	\caption{Protocol for random biased bit.}
% 	\label{prot:biasedbit}
% \end{protocol}
% \FloatBarrier

% % uniform distribution
% % ??? math formula needs to verify
% \subsubsection{Protocol for Uniform Distribution}
% Protocol $\Uppi^{Uniform\left(0,1 \right)}$~\cite{DBLP:journals/corr/WuHWX16} generating a share of a uniform random variable using the algorithm $Algo^{Uniform}$. Specifically, the uniform random variable is generated by inputing a Gaussian distribution function $g=\frac{1}{2}+\frac{1}{\sqrt{2\pi }}\int _{0}^{x}e^{-{t^{2}}}dt$ a Gaussian random variable $gau$ drawn from $g$, where $g$ is the distribution function of $Gaussian\left(0,1\right)$. To calculate the integral term of $g$, we can use composite trapezoidal method~\cite[Chapter 15]{ascher2011first}.
% \begin{protocol}[tbh!]
% 	\centering
% 	\fbox{
% 	\pseudocode[space=none, syntaxhighlight=auto, addkeywords={Protocol, Input, Output},linenumbering, skipfirstln, head=\textbf{Protocol: $\Uppi^{Uniform\left(0,1 \right)}$}]{
% 	\textbf{Input: $s=\left( s_{i1},\ldots ,s_{id}\right) _{i\in \left[ n\right] }\in \left( \left\{ 0,1\right\} ^{d}\right) ^{n}$ } \pcskipln \\
% 	\textbf{Output: ${\left( {\langle uni\rangle} _i \right)} _{i\in \left[ n\right] }$} \\
% 	\text{The parties choose a step length $h$ and step number $r$ such that $hr=1$.}\\
% 	\text{The parties run $\Uppi^{Gauss\left(0,\sigma^{2} \right)}$ with inputs $s$ and obtain ${ {\langle gau\rangle}}$ .} \\
% 	\text{Each party locally computes $\langle t_{k}\rangle =hk\langle gau\rangle $ for $k\in \left[ r\right] $. } \\
% 	\text{The parties compute $g\left( \langle t_{k}\rangle \right) $ for $k\in \left[ r\right] $, where $g\left( t\right) =\frac{1}{2} \frac{1}{\sqrt{2\pi }}e^{\left( -{t^{2}}\right) }$ .}\\
% 	\text{The parties compute $\langle uni\rangle =\frac{1}{2}+\frac{1}{2} h\langle gau\rangle \left( g\left( 0\right) +g\left( \langle gau\rangle \right) +2 \Sigma _{k=1}^{r-1} {g \left( \langle t_{k}\rangle \right)}\right) $ and output $\langle uni\rangle$.}
% 	}}
% 	\caption{Protocol for uniform distribution.}
% 	\label{prot:uniform}
% \end{protocol}
% \FloatBarrier

% \subsection{Protocol for Noise Generation}
% \subsubsection{Protocol for Gaussian Noise}
% \begin{protocol}[tbh!]
% 	\centering
% 	\fbox{
% 	\pseudocode[space=none, syntaxhighlight=auto, addkeywords={Protocol, Input, Output},linenumbering, skipfirstln, head=\textbf{Protocol: $\Uppi^{Gauss\left(0,\sigma^2 \right)}$}]{
% 	\textbf{Input: $s=\left( s_{i1},\ldots ,s_{id}\right) _{i\in \left[ n\right] }\in \left( \left\{ 0,1\right\} ^{d}\right) ^{n}$ } \pcskipln \\
% 	\textbf{Output: ${\left( {\langle gau\rangle} _i \right)} _{i\in \left[ n\right] }$} \\
% 	\text{The parties run $\Uppi^{RandBits}_d$ with inputs $s$ and obtain $\langle u_{l}\rangle $ for $l \in \left[ d\right]$.}\\
% 	\text{Each party locally computes ${\langle bino\rangle } = \sum ^{d}_{l=1}\langle u_{l}\rangle $.} \\
% 	\text{Each party locally computes ${ \langle gau\rangle}=\frac{\langle \sigma bino\rangle -\frac{d\sigma}{2}}{\frac{\sqrt{d}}{2}}$ and output ${ \langle gau\rangle}$. }
% 	}}
% 	\caption{Protocol for gaussian noise based on random bits.}
% 	\label{prot:gauss}
% \end{protocol}
% \FloatBarrier


% \subsubsection{Protocol for Laplace Noise}
% Further, based on $Algo^{Laplace}_{Binomial}$ and $Algo^{Laplace}_{Uniform}$, we could generate Laplace noise in MPC.

% \begin{protocol}[tbh!]
% 	\centering
% 	\fbox{
% 	\pseudocode[space=none, syntaxhighlight=auto, addkeywords={Protocol, Input, Output},linenumbering, skipfirstln, head=\textbf{Protocol: $\Uppi^{Lap\left(b=\frac{\sigma ^{2}}{2} \right)}_{Binomial}$}]{
% 	\textbf{Input: $s=\left( s_{it1},\ldots ,s_{itd}\right) _{i\in \left[ n\right] ,{t\in \left\{ 1,2,3,4\right\} }} \in \left( \left\{ 0,1\right\} ^{4\times d}\right) ^{n}$ } \pcskipln \\
% 	\textbf{Output: ${\left( {\langle lap\rangle} _{it} \right)} _{i\in \left[ n\right],{t\in \left\{ 1,2,3,4\right\} } }$} \\
% 	\text{The parties run $\Uppi^{Gauss\left(0,\sigma^{2} \right)}$ with inputs $\left( s_{it1},\ldots s_{itd}\right) _{i\in \left[ n\right] }$ and obtain $\langle gau_t\rangle $ for ${t\in \left\{ 1,2,3,4\right\} }$.} \\
% 	\text{The parties compute ${\langle lap\rangle } =\langle gau_1^2\rangle +\langle gau_2^2\rangle-\langle gau_3^2\rangle-\langle gau_4^2\rangle$ and output ${\langle lap\rangle }$.}
% 	}}
% 	\caption{Protocol for Laplace noise based on random bits.}
% 	\label{prot:randbitstolap}
% \end{protocol}
% \FloatBarrier

% \begin{protocol}[tbh!]
% 	\centering
% 	\fbox{
% 	\pseudocode[space=none, syntaxhighlight=auto, addkeywords={Protocol, Input, Output},linenumbering, skipfirstln, head=\textbf{Protocol: $\Uppi^{Lap\left(b \right)}$}]{
% 	\textbf{Input: $s=\left( s_{it1},\ldots ,s_{itd}\right) _{i\in \left[ n\right] ,{t\in \left\{ 1,2\right\} }} \in \left( \left\{ 0,1\right\} ^{2\times d}\right) ^{n}$ } \pcskipln \\
% 	\textbf{Output: ${\left( {\langle lap\rangle} \right)} _{i\in \left[ n\right] }$} \\
% 	\text{The parties run $\Uppi^{Uniform\left(0,1 \right)}$ with inputs $\left( s_{it1},\ldots s_{itd}\right) _{i\in \left[ n\right] }$ and obtain $\langle uni_t\rangle $ for ${t\in \left\{ 1,2\right\} }$.} \\
% 	\text{The parties compute ${\langle lap\rangle} =\langle b\left( \ln uni_{1} - \ln uni_{2} \right)\rangle $ and output ${\langle lap\rangle}$.}
% 	}}
% 	\caption{Protocol for Laplace noise based on uniform distribution.}
% 	\label{prot:uniformtolap}
% \end{protocol}
% \FloatBarrier

% \subsubsection{Protocol for Discrete Laplace Noise}

% We generate shares of discrete Laplace noise by using algorithm $Algo^{DLaplace}_{Uniform}$.
% \begin{protocol}[tbh!]
% 	\centering
% 	\fbox{
% 	\pseudocode[space=none, syntaxhighlight=auto, addkeywords={Protocol, Input, Output},linenumbering, skipfirstln, head=\textbf{Protocol: $\Uppi^{DLap\left(b \right)}$}]{
% 	\textbf{Input: $s=\left( s_{it1},\ldots ,s_{itd}\right) _{i\in \left[ n\right] ,{t\in \left\{ 1,2\right\} }} \in \left( \left\{ 0,1\right\} ^{2\times d}\right) ^{n}$ } \pcskipln \\
% 	\textbf{Output: ${\left( {\langle dlap\rangle} \right)} _{i\in \left[ n\right] }$} \\
% 	\text{The parties run $\Uppi^{Uniform\left(0,1 \right)}$ with inputs $\left( s_{it1},\ldots s_{itd}\right) _{i\in \left[ n\right] }$ and obtain $\langle u_t\rangle $ for ${t\in \left\{ 1,2\right\} }$.} \\
% 	\text{The parties compute $\langle dlap\rangle =\bigg\langle \bigg\lfloor \frac{1}{\ln b} \ln u_{1} \bigg\rfloor - \bigg\lfloor \frac{1}{\ln b} \ln u_{2} \bigg\rfloor\bigg\rangle $ and output $\langle dlap\rangle$.}
% 	}}
% 	\caption{Protocol for discrete laplace noise based on unifom distribution.}
% 	\label{prot:uniformtodlap}
% \end{protocol}
% \FloatBarrier

% % TODO: optimize \sigma MPC operation
% \subsubsection{Protocol for Finite-range Discrete Laplace Noise}
% \begin{protocol}[tbh!]
% 	\centering
% 	\fbox{
% 	\pseudocode[space=none, syntaxhighlight=auto, addkeywords={Protocol, Input, Output},linenumbering, skipfirstln, head=\textbf{Protocol: $\Uppi^{FDL\left(p,N \right)}_d$}]{
% 	\textbf{Input: $s=\left({\left( s_{i1}^{\left(j\right)},\ldots ,s_{id}^{\left(j\right)}\right) }_{j\in \left[ N-1\right] },\sigma_i \right)_{i\in \left[ n\right]} $, $\alpha =\left( \alpha_0,\ldots ,\alpha_{N-1}\right) $} \pcskipln \\
% 	\textbf{Output: $\left({ \langle k \rangle} _{i}\right) _{i\in \left[ n\right] } \sim FDL\left(p,N\right)$} \\
% 	\text{The parties run $\Uppi^{BiasedBit\left(\alpha_j \right)}$ with inputs $s_j=\left( s_{i1}^{\left(j\right)},\ldots s_{id}^{\left(j\right)}\right) _{i\in \left[ n\right] }$ and obtain $\langle b_{j}\rangle $ for $j\in \left[ N-1\right] $.} \\
% 	\text{The parties compute $ \langle c_{j}\rangle =\langle \vee_{k=1}^j b_{k}\rangle $ for $j\in \left[ N-1\right] $.} \\
% 	\text{Each party locally computes $\langle l\rangle =N-\sum ^{N-1}_{j=0}\langle c_j\rangle$.} \\
% 	\text{Each party shares $\sigma_i$ for $i \in \left[ n\right] $.}\\
% 	\text{The parties compute $\langle \sigma \rangle =\langle \prod _{i\in \left[ n\right] }\sigma_j\rangle $.}\\
% 	\text{The parties compute $\langle k\rangle =\langle \sigma l \rangle $ and output $\langle k\rangle$.}
% 	}}
% 	\caption{Protocol for Finite-range discrete laplace noise.}
% 	\label{prot:FDL}
% \end{protocol}
% \FloatBarrier