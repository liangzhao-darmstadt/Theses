\chapter{Fixed-Point and Floating-Point Operation}
\label{cha:FpFl}

Let $LT\left(a,b\right) $ be defined as follows:

\begin{equation}
    LT\left(a,b\right)=
    \begin{cases}
        1 \text{ if } a <b \\
        0 \text{ otherwise }
    \end{cases}
\end{equation}

Let $\left[x\right] _{2^{\ell}}$ denotes $x \mod 2^{\ell}$ and omit the subscript $2^{\ell}$ when it not affects the understanding.

For $a, b \in \mathbb{Z}_{2^{\ell}}$ (following arithmetic share is under $\mathbb{Z}_{2^{\ell}}$), we have $\left[a\right]=a $, $\left[b\right]=b $ and ~\cite{makri2021rabbit}:
\begin{equation}
    \label{eq:LT}
    \begin{split}
        \left[a +b\right]  &=\left[a\right]  + \left[b\right] -2^{\ell} \cdot LT\left(\left[a+b\right] ,\left[b\right] \right)   = \left[a\right] +\left[b\right]  -2^{\ell} \cdot LT\left(\left[a+b\right]  ,\left[a\right] \right) \\
        \left[ a -b\right]  &=\left[a\right]  -\left[ b\right]  +2^{\ell} \cdot LT\left(\left[a\right] ,\left[b\right] \right)
    \end{split}
\end{equation}
To extend SecFloat~\cite{rathee2022secfloat} to multiparty settings, we design following building blocks:

\begin{protocol}[tbh!]
    \centering
    \fbox{
    \pseudocode[space=none, syntaxhighlight=auto, addkeywords={Algorithm, Input, Output, IF,TO,RETURN, FOR, ELSE IF, ELSE, WHILE, TRUE, FALSE},linenumbering, skipfirstln, head=\textbf{Protocol: $\Uppi^{Wrap}\left(\left\langle x\right\rangle_0^A, \ldots ,\left\langle x\right\rangle_{\left(N-1\right)}^A  \right) $}]{
    \textbf{Input: Secret shared value $x$, such that parties hold $\left\langle x\right\rangle_0^A, \ldots ,\left\langle x\right\rangle_{\left(N-1\right)}^A$, where $x=\sum_{i=0}^{N-1} \left\langle x\right\rangle_i^A \mod 2^{\ell}  $} \pcskipln \\
    \textbf{Output: Compute the arithmetic secret share $\left\langle t\right\rangle ^A$, such that $\sum_{i=0}^{N-1} \left\langle x\right\rangle_i^A=x+t \cdot  2^{\ell}  $.} \\
    \text{Parties generate edaBit and each party holds $\left(\left\langle r\right\rangle^A_i, \left\langle r_0\right\rangle^B_i,\ldots,  \left\langle r_{\ell-1}\right\rangle^B_i \right) $, where $ r=\sum_{j=0}^{\ell-1} 2^{j}\cdot  r_j  $.}\\
    \text{Each party locally computes $t_i^{\left(1\right) }=LT\left(\left[\left\langle x\right\rangle ^A_i +\left\langle r \right\rangle^A_i \right] ,\left\langle r\right\rangle^A_i\right) $ by taking $ \left\langle x\right\rangle ^A_i   $ and $\left\langle r\right\rangle^A_i$ as plaintext values.}\\
    \text{Each party broadcasts the computed $ \left[\left\langle x\right\rangle ^A_i+\left\langle r\right\rangle^A_i \right]  $ to other parties.}\\
    \text{Each party locally computes the publicly known $t^{(2)}$, where $\sum_{i = 0}^{N-1}\left[\left\langle x\right\rangle ^A_i+\left\langle r\right\rangle^A_i \right] = \left[x+r \right]   + t^{(2)}\cdot 2^{\ell} $.}\\
    \text{Parties compute $\left\langle t^{(3)}\right\rangle^A$ using $\Uppi_{BitAdder}$, where $\sum_{i = 0}^{N-1} \left\langle r\right\rangle ^A_i=r+  t^{(3)} \cdot 2^{\ell} $.}\\
    \text{Parties compute $\left\langle t^{(4)}\right\rangle ^A=\Uppi_{LT}\left(\left[x+r\right] , \left\langle r\right\rangle^A  \right) $.}\\
    \text{Parties set $\left\langle t\right\rangle^A_i =\left[t^{(1)}+t^{\left(2\right) }_i-\left\langle t^{(3)}\right\rangle^A_i-\left\langle t^{(4)}\right\rangle^A_i \right] $ as an arithmetic share of $t$.}
    }}
    \caption{Protocol for wrap operation.}
    \label{prot:wrap}
\end{protocol}
\FloatBarrier

\section{Intuition of $\Uppi^{Wrap}\left(\left\langle x\right\rangle_0^A, \ldots ,\left\langle x\right\rangle_{\left(N-1\right)}^A  \right) $}

For $i \in \left[0,N-1\right] $,
suppose that each party $i$ holds an arithmetic share $\left\langle x\right\rangle_i^A$ of a secret $x$ (i.e., $ x=\left[\sum_{i=0}^{N-1} \left\langle x\right\rangle_i^A \right]  _{2^\ell} $ and $ \left\langle x\right\rangle_i^A \in \mathbb{Z} _{2^{\ell}}$), and edaBit~\cite{escudero2020improved} $\left\langle r\right\rangle^A_i, \left\langle r_0\right\rangle^B_i,\ldots,  \left\langle r_{\ell-1}\right\rangle^B_i $ (i.e., $r=\sum_{j=0}^{\ell-1} 2^{j}\cdot r_j  $ and $r \in \mathbb{Z} _{2^{\ell}}$), we want to compute an arithmetic share $\left\langle t\right\rangle^A $ of $t$, where
\begin{equation}
    \label{eq:wrap}
    x + t\cdot 2^{\ell}   =\left\langle x\right\rangle_0^A+ \ldots +\left\langle x\right\rangle_{\left(N-1\right)}^A
\end{equation}
One straighforward way to compute $\left\langle t\right\rangle ^A$ is to first convert $\left\langle x\right\rangle_{i}^A$ into Boolean shares $\left\langle x_0\right\rangle_{i}^B,\ldots, \left\langle x_{\ell-1}\right\rangle_{i}^B$, where $\left\langle x\right\rangle_{i}^A=\left[\sum_{j = 0}^{\ell-1} 2^{j}\cdot \left\langle x_{j}\right\rangle_{i}^B\right]  _{2^{\ell}} $. Then, each party $i$ input $\left\langle x_0\right\rangle_{i}^B,\ldots, \left\langle x_{\ell-1}\right\rangle_{i}^B$ as private values into an adder circuit (with $\ell+\log_2 N$ output bits) and only obtain the first $\log_2 N$ output bits and convert it into arithmetic shares to get $\left\langle t\right\rangle ^A$. To avoid expensive online computation, we use edaBit (mask-and-open technique) and \autoref{eq:LT} to shift computations into offline phase and make the online phase more efficient.

\CHANGED{Recalls that for arithmetic sharing, linear operations can be computed locally without communication (i.e., $\left\langle t\right\rangle^A_i =t^{pub} + \left\langle t^{priv}\right\rangle ^A_i$ for public value $t^{pub}$ and secret share $\left\langle t^{priv}\right\rangle ^A_i$). The idea is to split $\left\langle t\right\rangle^A_i$ into a public value $t^{pub}$ (cf.~\autoref{wrap:step2}) and secret shares $\left\langle t^{priv}\right\rangle ^A_i$ (cf.~\autoref{wrap:step1}, ~\autoref{wrap:step3}, ~\autoref{wrap:step4}), and to reduce online-communication. The computation of $t^{pub}$ only requires one round of online-communication (e.g., similar to reconstruction an secret value where eacy party broadcast the values they hold). We further divide $\left\langle t^{priv}\right\rangle ^A_i$ into two parts: $\left\langle t^{priv-offline}\right\rangle ^A_i$ (cf.~\autoref{wrap:step1}, ~\autoref{wrap:step3}) and $\left\langle t^{priv-online}\right\rangle ^A_i$ (cf.~\autoref{wrap:step4}). $\left\langle t^{priv-offline}\right\rangle ^A_i$ can be precomputed by the parties and require no online-communication. $\left\langle t^{priv-online}\right\rangle ^A_i$ is the major overhead of online-communication as it requires protocol $\Uppi^{LT}$~\cite{makri2021rabbit}. Using edaBit~\cite{escudero2020improved} and \autoref{eq:LT}, we try to move the most part of the computation of $\left\langle t\right\rangle^A_i $ into the computation of $\left\langle t^{priv-offline}\right\rangle ^A_i$. More specifically, each party starts by masking their local share $\left\langle x\right\rangle_{i}^A$ with the randomly generated share $\left\langle r\right\rangle_{i}^A$ to keep the $x$ secret and obtain $\left\langle x\right\rangle_{i}^A=\left(\left\langle x\right\rangle_{i}^A+\left\langle r\right\rangle_{i}^A\right)-\left\langle r\right\rangle_{i}^A $. Then, by using \autoref{eq:LT} intensively, we reveal some intemediate values and finally get $\left\langle x\right\rangle_0^A+\ldots +\left\langle x\right\rangle_{\left(N-1\right)}^A=\left[\sum_{i=0}^{N-1} \left\langle x\right\rangle_i^A \right]_{2^{\ell}} +\left[\sum_{i=0}^{N-1} \left\langle t\right\rangle_i^A \right]_{2^{\ell}}\cdot 2^{\ell}$, where each party holds $\left\langle x\right\rangle_i^A$ and $\left\langle t\right\rangle_i^A$. Note that the overall online-communication complexity is consist of two parts: broadcasting (cf. ~\autoref{wrap:step2}) and $\Uppi^{LT}$(cf. ~\autoref{wrap:step4}: one invocation of $\Uppi^{LT}$).}


\section{Correctness of $\Uppi^{Wrap}\left(\left\langle x\right\rangle_0^A, \ldots ,\left\langle x\right\rangle_{\left(N-1\right)}^A  \right) $}

In following, we compute $\left\langle t\right\rangle^A_i =\left[t^{(1)}_i+t^{\left(2\right) }-\left\langle t^{(3)}\right\rangle^A_i-\left\langle t^{(4)}\right\rangle^A_i \right] $ in four steps, that is defined as follows:
\begin{equation}
    \begin{split}
        t^{pub}&: t^{(2)} \\
        \left\langle t^{priv-offline}\right\rangle ^A_i&: t^{(1)}_i,\left\langle t^{(3)}\right\rangle^A_i \\
        \left\langle t^{priv-online}\right\rangle ^A_i&: \left\langle t^{(4)}\right\rangle^A_i
    \end{split}
\end{equation}

And prove the correctness of $t=\sum_{i=0}^{N-1} t^{(1)}_i+t^{(2)}  -  t^{\left(3\right)} -t^{\left(4\right) } $.

\subsection{Step 1. Compute $t^{(1)}_i=LT\left(\left[\left\langle x\right\rangle_i^A + \left\langle r\right\rangle^A_i\right] ,\left\langle r\right\rangle^A_i \right)$}
\label{wrap:step1}

First, as $ \left[a +b\right]  =\left[a\right]  + \left[b\right] -2^{\ell} \cdot LT\left(\left[a+b\right] ,\left[b\right] \right) $ (cf.~\autoref{eq:LT}), we replace $a$ with $\left\langle x\right\rangle_i^A$ and $b$ with $\left\langle r\right\rangle_i^A$ and get follows:
\begin{equation}
    \begin{split}
        \left[\left\langle x\right\rangle_i^A + \left\langle r\right\rangle^A_i\right] =\left\langle x\right\rangle_i^A + \left\langle r\right\rangle^A_i-2^{\ell}\cdot LT\left(\left[\left\langle x\right\rangle_i^A + \left\langle r\right\rangle^A_i\right] ,\left\langle r\right\rangle^A_i \right)\\
    \end{split}
\end{equation}

which can be reformulated as follows:
\begin{equation}
    \label{eq:LT2}
    \begin{split}
        \left\langle x\right\rangle_i^A=\left[\left\langle x\right\rangle_i^A + \left\langle r\right\rangle^A_i\right] -\left\langle r\right\rangle^A_i+2^{\ell}\cdot LT\left(\left[\left\langle x\right\rangle_i^A + \left\langle r\right\rangle^A_i\right] ,\left\langle r\right\rangle^A_i \right)
    \end{split}
\end{equation}


Note that $t^{(1)}_i=LT\left(\left[\left\langle x\right\rangle_i^A + \left\langle r\right\rangle^A_i\right] ,\left\langle r\right\rangle^A_i \right)$ is computed by each party $i$ locally, i.e., takes the values of $\left\langle x\right\rangle_i^A $ and $\left\langle r\right\rangle^A_i$, calculate the modulo $\left[\left\langle x\right\rangle_i^A + \left\langle r\right\rangle^A_i\right] $ and compare it with $\left\langle r\right\rangle^A_i$ in plaintext. Note that $t^{(1)}_i \in \left\{0,1\right\} $ and is only known to party $i$.

Then, we can rewrite~\autoref{eq:wrap} with~\autoref{eq:LT2} as follows:
\begin{equation}
    \begin{split}
        \label{eq:wrap1}
        x + t\cdot 2^{\ell}  & =\left\langle x\right\rangle_0^A+ \ldots +\left\langle x\right\rangle_{\left(N-1\right)}^A \\
        &=\left[ \left\langle x\right\rangle_0^A+ \left\langle r\right\rangle^A_0 \right]   -\left\langle r\right\rangle^A_0+ 2^{\ell}\cdot LT\left(\left[\left\langle x\right\rangle_0^A+ \left\langle r\right\rangle^A_0 \right]  ,\left\langle r\right\rangle^A_0\right)
        + \ldots \\
        &= \sum_{i = 0}^{N-1} \left[\left\langle x\right\rangle_i^A+ \left\langle r\right\rangle^A_i\right] -\sum_{i = 0}^{N-1}\left\langle r\right\rangle^A_i+ 2^{\ell}\cdot \sum_{i = 0}^{N-1}LT\left(\left[\left\langle x\right\rangle_i^A+ \left\langle r\right\rangle^A_i  \right] ,\left\langle r\right\rangle^A_i\right)\\
        &= \sum_{i = 0}^{N-1}  \left[ \left\langle x\right\rangle_i^A+ \left\langle r\right\rangle^A_i\right]   -\sum_{i = 0}^{N-1}\left\langle r\right\rangle^A_i+ 2^{\ell}\cdot \sum_{i = 0}^{N-1}t^{(1)}_i
    \end{split}
\end{equation}
Next, we explain how to compute the first two terms $ \sum_{i = 0}^{N-1} \left[\left\langle x\right\rangle_i^A+ \left\langle r\right\rangle^A_i \right] $ and $\sum_{i = 0}^{N-1}\left\langle r\right\rangle^A_i$.


\subsection{Step 2. Compute $t^{(2)}$, where $ \sum_{i = 0}^{N-1}  \left[ \left\langle x\right\rangle_i^A+ \left\langle r\right\rangle^A_i \right]  =\left[x+r  \right]   + t^{(2)}\cdot 2^{\ell}$}
\label{wrap:step2}


Each party can locally compute $ \left[\left\langle x\right\rangle ^A_i+\left\langle r\right\rangle^A_i\right]    $ by taking $\left\langle x\right\rangle ^A_i$ and $\left\langle r\right\rangle^A_i$ as plaintext values, compute the addition and modulo operation, and broadcast it to other parties. Then all parties use the received values to compute $\sum_{i = 0}^{N-1}\left[ \left\langle x\right\rangle ^A_i+\left\langle r\right\rangle^A_i  \right]  = \left[x+r  \right]   + t^{(2)}\cdot 2^{\ell} $ in plaintext, where $t^{(2)}$ is an integer. Note that both $ \left[x+r \right]  $ and $t^{(2)}$ are publicly known, but as $r$ is a random value, $ \left[x+r \right]  $ reveals no information about $x$.


\subsection{Step 3. Compute $\left\langle t^{\left(3\right) }\right\rangle^A$, where $\sum_{i = 0}^{N-1}\left\langle r\right\rangle^A_i=r+  t^{\left(3\right) } \cdot 2^{\ell}$}
\label{wrap:step3}

For $\sum_{i = 0}^{N-1}\left\langle r\right\rangle^A_i$, the parties can deploy an adder circuit. More specifically, each party $i$ with private input $ \left\langle r_0\right\rangle^B_i,\ldots,  \left\langle r_{\ell-1}\right\rangle^B_{i} $ computes and obtains $ \left\langle t^{\left(3\right) }\right\rangle^A$ together, where $\sum_{i = 0}^{N-1} \left\langle r \right\rangle ^A_i=r+   t^{\left(3\right) }  \cdot 2^{\ell} $ and $ \left\langle t^{\left(3\right) }\right\rangle^A$ are computed by converting the corresponding output bits of the adder circuit without revealing $r$. Note that this step can be precomputed.

Next, we transform~\autoref{eq:wrap1} as follows:
\begin{equation}
    \label{eq:wrap2}
    \begin{split}
        x + t\cdot 2^{\ell}  &= \sum_{i = 0}^{N-1}   \left[\left\langle x\right\rangle_i^A+ \left\langle r\right\rangle^A_i \right]  -\sum_{i = 0}^{N-1}\left\langle r\right\rangle^A_i+ 2^{\ell}\cdot \sum_{i = 0}^{N-1}t^{(1)}_i\\
        &= \left[x+r\right]     + t^{(2)}\cdot 2^{\ell}  -\left(r+   t^{(3)}  \cdot 2^{\ell} \right) + 2^{\ell}\cdot \sum_{i = 0}^{N-1}t^{(1)}_i\\
        &= \left[x+r\right]    -r +2^{\ell}\cdot \left(t^{(2)}  -  t^{(3)} +\sum_{i=0}^{N-1} t^{(1)}_i\right) \\
    \end{split}
\end{equation}
As $ \left[ a -b\right]   =\left[a\right]  -\left[ b\right]  +2^{\ell} \cdot LT\left(\left[a\right] ,\left[b\right] \right)$ (cf.~\autoref{eq:LT}), when we replace $a$ with $ \left[x+r  \right] $ and $b$ with $r$, we get as follows:
\begin{equation}
    \begin{split}
        \left[ \left[x+r\right]    -r  \right]  = \left[x+r  \right]  - r+2^{\ell} \cdot LT\left( \left[x+r \right] ,r\right)
    \end{split}
\end{equation}
which can be transformed into:
\begin{equation}
    \begin{split}
        \left[x+r\right]   - r=     \left[ \left[x+r\right]   -r \right]     -2^{\ell} \cdot LT\left( \left[x+r  \right] ,r\right)
    \end{split}
\end{equation}

Next, we take above equation into~\autoref{eq:wrap2} and get as follows:
\begin{equation}
    \label{eq:wrap3}
    \begin{split}
        x + t\cdot 2^{\ell}  &=\left[x+r\right]  -r +2^{\ell}\cdot \left(t^{(2)}  - t^{\left(3\right) }+\sum_{i=0}^{N-1} t^{(1)}_i\right) \\
        &=  \left[ \left[x+r\right]   -r \right]     -2^{\ell} \cdot LT\left( \left[x+r  \right] ,r\right)+2^{\ell}\cdot \left(t^{(2)}  -  t^{\left(3\right) } +\sum_{i=0}^{N-1} t^{(1)}_i\right)
    \end{split}
\end{equation}

\subsection{Step 4. Compute $\left\langle t^{\left(4\right) }\right\rangle^A $, where $t^{\left(4\right) }=LT\left(\left[x+r  \right] ,  r  \right) $}
\label{wrap:step4}

As $\left[ \left[x+r\right]   -r \right]=x$, \autoref{eq:wrap3} can be transformed as follows:
\begin{equation}
    \label{eq:wrap4}
    \begin{split}
        x + t\cdot 2^{\ell} &=  \left[ \left[x+r\right]   -r \right]     -2^{\ell} \cdot LT\left( \left[x+r  \right] ,r\right)+2^{\ell}\cdot \left(t^{(2)}  -  t^{\left(3\right) } +\sum_{i=0}^{N-1} t^{(1)}_i\right)\\
        &= x-2^{\ell} \cdot t^{\left(4\right) }+2^{\ell}\cdot \left(t^{(2)}  -  t^{\left(3\right) } +\sum_{i=0}^{N-1} t^{(1)}_i\right)\\
        &= x +2^{\ell}\cdot \left(\sum_{i=0}^{N-1} t^{(1)}_i+t^{(2)}  -  t^{\left(3\right)} -t^{\left(4\right) } \right)
    \end{split}
\end{equation}

The parties compute and obtain $\left\langle t^{\left(4\right) }\right\rangle^A $ together, where $t^{\left(4\right) }=LT\left(\left[x+r  \right] ,r\right) $. Note that $\left[x+r  \right] $ is publicly known and $r$ is secret shared as $\left\langle r\right\rangle^A $. Therefore, we use the $\Uppi^{LT}$~\cite{makri2021rabbit} to compare the publicly known $ \left[x+r  \right] $ with arithmetic share $\left\langle r\right\rangle ^A $ of $r$ and convert the comparison result into arithmetic share $\left\langle t^{\left(4\right) }\right\rangle^A$.

Finally, each party $i$ set $\left\langle t\right\rangle^A_i =\left[t^{(1)}_i+t^{\left(2\right) }-\left\langle t^{(3)}\right\rangle^A_i-\left\langle t^{(4)}\right\rangle^A_i \right] $ as an arithmetic share of $t$, where $t^{(2)}$ is publicly known and $\left(t_i^{\left(1\right) },\left\langle t^{(3)}\right\rangle^A_i,\left\langle t^{(4)}\right\rangle^A_i\right) $ are only known to party $i$.


\section{Toy Example of $\Uppi^{Wrap}\left(\left\langle x\right\rangle_0^A, \ldots ,\left\langle x\right\rangle_{\left(N-1\right)}^A  \right) $}
Suppose $x=2$ and $r=6$, and we operate under field $\mathbb{Z} _{2^{3}}$ with three parties as follows:

$P_0$: $\left\langle x\right\rangle^A_0=7 $, $\left\langle r\right\rangle^A_0=3 $, $\left\langle r\right\rangle ^B_0=\left(0,1,1\right) $

$P_1$: $\left\langle x\right\rangle^A_1=6 $, $\left\langle r\right\rangle^A_1=4 $, $\left\langle r\right\rangle ^B_1=\left(1,0,0\right) $

$P_2$: $\left\langle x\right\rangle^A_2=5 $, $\left\langle r\right\rangle^A_2=7 $, $\left\langle r\right\rangle ^B_2=\left(1,1,1\right) $

Then, we have $\left\langle x\right\rangle^A_0+\left\langle x\right\rangle^A_1+\left\langle x\right\rangle^A_2=x+t\times 2^{3}=7+6+5=18=2+2\times 2^{3}$ and $t=2$.
\\

\textbf{1. Compute $t^{(1)}_i=LT\left(\left[\left\langle x\right\rangle_i^A + \left\langle r\right\rangle^A_i\right] ,\left\langle r\right\rangle^A_i \right)$}

$P_0$: $t^{(1)}_0=LT\left(\left[\left\langle x\right\rangle_0^A + \left\langle r\right\rangle^A_0\right] ,\left\langle r\right\rangle^A_0 \right)=\left(\left[7+3\right] <3\right)=\left(2 <3\right) =1$

$P_1$: $t^{(1)}_1=LT\left(\left[\left\langle x\right\rangle_1^A + \left\langle r\right\rangle^A_1\right] ,\left\langle r\right\rangle^A_1 \right)=\left(\left[6+4\right] <4\right)=\left(2 <4\right) =1$

$P_2$: $t^{(1)}_2=LT\left(\left[\left\langle x\right\rangle_2^A + \left\langle r\right\rangle^A_2\right] ,\left\langle r\right\rangle^A_2 \right)=\left(\left[5+7\right] <7\right)=\left(4 <7\right) =1$\\

\textbf{2. Compute $t^{(2)}$, where $ \sum_{i = 0}^{2}  \left[ \left\langle x\right\rangle_i^A+ \left\langle r\right\rangle^A_i \right]  =\left[x+r  \right]   + t^{(2)}\cdot 2^{3}$}

$P_0$, $P_1$, $P_2$: $\sum_{i = 0}^{2}  \left[ \left\langle x\right\rangle_i^A+ \left\langle r\right\rangle^A_i \right]=\left[7+3\right] +\left[6+4\right]+\left[5+7\right] =2+2+4=8=0+1\times 2^{3}$

We get publicly known values: $\left[x+r  \right]=0$, $t^{(2)}=1$\\

\textbf{3. Compute $\left\langle t^{\left(3\right) }\right\rangle^A$, where $\sum_{i = 0}^{2}\left\langle r\right\rangle^A_i=r+  t^{\left(3\right) } \cdot 2^{3}$}

$P_0$, $P_1$, $P_2$: $\Uppi^{AddCircuit}\left(\left\langle r\right\rangle ^B_0, \left\langle r\right\rangle ^B_1,\left\langle r\right\rangle ^B_2\right) =\left(0,1,1\right)+\left(1,0,0\right)+\left(1,1,1\right) =\left(0,1,1,1,0\right) $

The parties only obtain the Boolean sharing of the first two (i.e., $\left\lceil \log_2 3\right\rceil $) output bits $\left(0,1\right) $ and convert it into arithmetic shares $\left\langle t^{\left(3\right) }\right\rangle^A_i$, where $\left[\sum_{i=0}^{2} \left\langle t^{\left(3\right) }\right\rangle^A_i\right] _{2^3}=\left(0,1\right)_2=1_{10} $ (where $\left(\cdot\right)_2 $ and $\left(\cdot\right)_{10} $ denote the binary and decimal representation).
Let's suppose parties get shares $\left\langle t^{\left(3\right) }\right\rangle^A_i$ as follows:

$P_0$: $\left\langle t^{\left(3\right) }\right\rangle^A_0=5$

$P_1$: $\left\langle t^{\left(3\right) }\right\rangle^A_1=7$

$P_2$: $\left\langle t^{\left(3\right) }\right\rangle^A_2=5$

\textbf{4. Compute $\left\langle t^{\left(4\right) }\right\rangle^A $, where $t^{\left(4\right) }=LT\left(\left[x+r  \right] ,  r  \right) $}

As $\left[x+r  \right] =0$, we deploy $\Uppi^{LT}$ to compare publicly known value $\left[x+r  \right] =0$ with secret $r$.
Suppose after comparison, each party get $\left\langle t^{\left(4\right) }\right\rangle^A_i $ (i.e., $\left[\sum_{i=0}^{2}  \left\langle t^{\left(4\right) }\right\rangle^A_i\right] _{2^{3}} =\left(\left[x+r  \right]<r\right) =1$) as follows:

$P_0$: $\left\langle t^{\left(4\right) }\right\rangle^A_0 =3$

$P_1$: $\left\langle t^{\left(4\right) }\right\rangle^A_1= 7$

$P_2$: $\left\langle t^{\left(4\right) }\right\rangle^A_1= 7$

Finally, parties set their arithmetic share $\left\langle t\right\rangle^A_i =\left[t^{(1)}_i+t^{\left(2\right) }-\left\langle t^{(3)}\right\rangle^A_i-\left\langle t^{(4)}\right\rangle^A_i \right] $:

$P_0$: $\left\langle t\right\rangle^A_0= \left[t^{(1)}_0+t^{\left(2\right) }-\left\langle t^{(3)}\right\rangle^A_0-\left\langle t^{(4)}\right\rangle^A_0\right] =\left[1+1-5-3\right] =2$

$P_1$: $\left\langle t\right\rangle^A_1= \left[t^{(1)}_1-\left\langle t^{(3)}\right\rangle^A_1-\left\langle t^{(4)}\right\rangle^A_1\right] =\left[1-7-7\right] =3$

$P_2$: $\left\langle t\right\rangle^A_2= \left[t^{(1)}_2-\left\langle t^{(3)}\right\rangle^A_2-\left\langle t^{(4)}\right\rangle^A_2\right]   =\left[1-5-7\right]=5 $

Note that as $t^{\left(2\right) }$ is publicly known, we only involve that in the computation of $\left\langle t\right\rangle^A_0$.

It is easy to verify $\left[\left\langle t\right\rangle^A_0+\left\langle t\right\rangle^A_1+\left\langle t\right\rangle^A_2\right]=\left[2+3+5\right] =2 $, which equals to $t=2$.